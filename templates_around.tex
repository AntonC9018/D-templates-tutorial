\newpage %%%%%%%%%%%%%%%%%%%%%%%%%%%%%%%%%%%%%%%%%%%%%%%%%%%%%%%%%%%%%%%%%%%%%%%
\part{Around Templates: Other Compile-Time Tools}\label{around}

There is more to compile-time metaprogramming in D than \emph{just} templates. This part will describe the most common tools: string mixins (\ref{stringmixins}), compile-time function evaluation (\ref{ctfe}) and \D{\_\_traits} (\ref{traits}), as seen in relation with templates. For the good news is: they are all interoperable. String mixins are wonderful to inject code in your templates, compile-time-evaluable functions can act as template parameters and can be templated. And, best of best, templated compile-time functions can return strings which can in turn be mixed-in\ldots in your templates. Come and see, it's fun!

\section{String Mixins} \label{stringmixins}

String mixins\index{string mixins} put D code were they are called, just before compilation. Once injected, the code is \emph{bona fide} D code, like any other. Code is manipulated as strings, hence the name.

\subsection{Syntax}\label{stringmixinssyntax}

The syntax is slightly different from mixin templates (\ref{mixintemplates}):

\index{syntax!string mixins}
\begin{dcode}
mixin("some code as a string");
\end{dcode}

You must take care not to forget the parenthesis. String mixins are a purely compile-time tool, so the string must also be determined at compile-time.

\subsection{Mixing Code In, With Templates}\label{stringmixinsandtemplates}

Of course, just injecting predefined code is a bit boring:

\index{example!string mixins}
\begin{dcode}
mixin("int i = 3;"); // Do not forget the two semicolons
                     // one for the mixed-in code,
                     // one for the mixin() call.
i++;
assert(i == 4);
\end{dcode}

There is no interest in that compared to directly writing standard D code. The fun begins with D powerful constant folding ability: in D, strings can be concatenated at compile-time.\index{compile-time!string concatenation} That's where string mixins meet templates: templates can produce strings at compile-time and can get strings as parameters. You alreday saw that in section \ref{operatoroverloading} on operator overloading and section \ref{opdispatch} on \DD{opDispatch}, since I couldn't help doing a bit of foreshadowing.

Now, imagine for example wanting a template that generates structs for you. You want to be able to name the structs as you wish. Say we would like the usage to look like that:

\begin{dcode}
module mine;
import named;

mixin(Named!"First");  // creates struct First { ... }
mixin(Named!"Second"); // and struct Second { ... }

First f1, f2;
Second s1;

assert(is( typeof(s1) == mine.First));
\end{dcode}

Here comes the generating code:

\index{example!string mixins}
\begin{dcode}
module named;

template Named(string name)
{
    enum string Named = "struct " ~ name ~ " { "
                      ~ "/+ some code +/"
                      ~ " }";
}

/* For example, name == "First" ->
   struct First { /+ some code +/ }
*/
\end{dcode}

In this case, the string is assembled inside the template during instantiation,  exposed through the eponymous trick\index{eponymous trick!string mixins} and then mixed in where you want it. Note that the string is generated in the module containing \DD{Named}, but that \DD{First} and \DD{Second} are defined exactly where the \D{mixin}\DD{()} call is. If you use the mixin in different modules, this will define as many different structs, all named the same way. This might be exactly what you want, or not.

To get the same struct in different modules, the code must be organized a bit differently: the structs must be generated in the template module (for example):

\index{example!string mixins}
\begin{dcode}
module named;

template Named(string name)
{
    alias NamedImpl!(name).result Named;
}

template NamedImpl(string name)
{    
    enum string Named = "struct " ~ name ~ " { "
                      ~ "/+ some code +/"
                      ~ " }";
    mixin(Named);
    mixin("alias " ~ name ~ " result;");
}
\end{dcode}

\begin{dcode}
module mine;
import named;

named!"First" f1, f2;
named!"Second" s1;
\end{dcode}

Usage is a different, as you can see. In this case, \DD{First} is generated inside \DD{NamedImpl} and exposed through an alias (this particular alias statement is itself generated by a string mixin). In fact, the entire code could be put in the mixin:

\begin{dcode}
module named;

template Named(string name)
{
    alias NamedImpl!(name).result Named;
}

template NamedImpl(string name)
{
mixin("struct " ~ name ~ " {"
                ~ "/* some code */"
                ~ " }\n"
   ~ "alias " ~ name ~ " result;");
}
\end{dcode}

Here is an example using the ternary \DD{?:} operator to do some compile-time selection of code, similar to what can be done with \D{static if} (\ref{staticif}):

\index{example!string mixins}
\begin{dcode}
enum GetSet { no, yes}

struct S(GetSet getset = GetSet.no, T)
{
    enum priv = "private T value;\n"
               ~ "T get() @property { return value;}\n"
               ~ "void set(T _value) { value = _value;}";

    enum pub = "T value;";

    mixin( (getset == GetSet.yes) ? priv : pub);
}

S!(GetSet.yes, int) gs;

/* Generates:

struct S!(GetSet.yes, int)
{
    private int value;
    int get() @property { return value;}
    void set(int _value) { value = _value;}
}
*/

gs.set(1);
assert( gs.get == 1);
\end{dcode}

\subsection{Limitations}\label{stringmixinslimitations}

Code crafting is still a bit awkward\index{string mixins!limitations}, because I haven't introduced CTFE yet (see \ref{ctfe}). So we are limited to simple concatenation for now: looping for example is possible with templates, but far easier with CTFE. Even then, it's already wonderfully powerful: you can craft D code with some `holes' (types, names, whatever) that will be completed by a template instantiation and then mixed in elsewhere. You can create other any kind of D code with that.

You can put \D{mixin}\DD{()} expressions almost were you want to, but\ldots

\TODO{Test the limits:inside static if expressions, for example}


\aparte{Escaping strings}{One usual problem with manipulating D code as string is how to deal with strings in code? You must escape them. Either use \DD{\textbackslash"} to create string quotes, a bit like was done in section \ref{functiontemplatessyntax} to generate the error message for \DD{select}. Or you can put strings between \DD{q\{} and \DD{\}}. }

\section{Compile-Time Function Evaluation} \label{ctfe}

\subsection{Evaluation at Compile-Time} \label{compiletimeevaluation}

Compile-Time Function Evaluation\index{CTFE}\index{Compile-Time Function Evaluation} (from now on, CTFE) is an extension of the constant-folding that's done during compilation\index{compile-time!constant folding} in D code: if you can calculate \DD{1 + 2 + 3*4} at compile-time, why not extend it to whole functions evaluation? I'll call evaluable at compile-time functions CTE functions from now on.

It's a very hot topic in D right now and the reference compiler has advanced by leaps and bounds in 2011. The limits to what can be done with CTE functions are pushed farther away at each new release.  All the \D{foreach}, \D{while}, \D{if}/\D{then}/\D{else} statements, arrays manipulation, struct manipulation, function manipulation\ldots are there. You can even do pointer arithmetics! When I began this document (DMD 2.055), the limitations\index{CTFE!limitations} were mostly: no classes and no exceptions (and so, no \DD{enforce}). This was changed with DMD 2.057, allowing the manipulation of classes at compile-time.

In fact danger lies the other way round: it's easy to forget that CTE functions must also be standard, runtime, functions. Remember that some actions only make sense at compile-time or with compile-time initialized constants: indexing on tuples for example:

\subsection{\D{\_\_ctfe}}

\unfinished{Write something on this new functionality, which enables testing inside a function whether we are at compile-time or runtime.}

\subsection{Templates and CTFE} \label{templatesandctfe}
\index{CTFE!and templates}
\index{template!and CTFE}

\unfinished{Some juicy examples should be added.}

That means: you can feed compile-time constants to your classical D function and its code will be evaluated at compile-time. As far as templates are concerned, this means that function return values can be used as template parameters and as \D{enum} initializers:

\index{example!CTFE}
\begin{dcode}

\end{dcode}

Template functions can very well give rise to functions evaluated at compile-time:

\begin{dcode}
\end{dcode}

\subsection{Templates and CTFE and String Mixins, oh my!}
\label{templatesandctfeandstringmixins}

And the fireworks is when you mix (!) that with string mixins: code can be generated by functions, giving access to almost the entire D language to craft it. This code can be mixed in templates to produce what you want. And, to close the loop: the function returning the code-as-string can itself be a template, using another template parameters as its own parameters.

Concretly, here is the getting-setting code from section \ref{stringmixinsandtemplates}, reloaded:

\begin{dcode}
enum GetSet { no, yes}

string priv(string type, string index)
{
    return 
    "private "~type~"value"~index~";\n"
  ~ type~" get"~index~"() @property { return value"~index~";}\n"
  ~ "void set"~index~"("~type~" _value) { value"~index~" = _value;}";
}   

string pub(string type, string index)
{
    return type ~ "value" ~ index ~ ";";
}

string GenerateS(GetSet getset = GetSet.no, T...)()
{
    string result;    
    foreach(index, Type; T)
        static if (getset = GetSet.yes)
		     result ~= priv(Type.stringof, to!string(index));
        else
            result ~= pub(Type.stringof, to!string(index));
    return result;
}

struct S(GetSet getset = GetSet.no, T...)
{
    mixin(GenerateS!(getset,T));
}

S!(GetSet.yes, int, string, int) gs;
/* Generates:

struct S!(GetSet.yes, int, string, int)
{
    private int value0;
    int get0() @property { return value0;}
    void set0(int _value) { value0 = _value;}

    private string value1;
    string get1() @property { return value1;}
    void set1(string _value) { value1 = _value;}

    private int value2;
    int get2() @property { return value2;}
    void set2(int _value) { value2 = _value;}
}
*/

gs.set1("abc");
assert(gs.get1 == "abc");
\end{dcode}

This code is much more powerful than the one we saw in section \ref{stringmixinsandtemplates}: the number of types is flexible, and an entire set of getters/setters is generated when asked to. All this is done by simply plugging \D{string}-returning functions together, and a bit of looping by way of a compile-time \D{foreach}.

\subsection{Simple String Interpolation}

All this play with the concatenating operator (\DD{\~}) is becoming a drag. We should write a string interpolation function, evaluable at compile-time of course, to help us in our task. Here is how I want to use it:

\begin{dcode}
alias interpolate!"struct #0 { #1 value; #0[#2] children;}" makeTree;

enum string intTree = makeTree("IntTree", "int", 2);
enum string doubleTree = makeTree("DoubleTree", "double", "");

assert(intTree 
       == "struct IntTree { int value; IntTree[2] children;}");
assert(doubleTree 
       == "struct DoubleTree { double value; IntTree[] children;}");
\end{dcode}

As you can see, the string to be interpolated is passed as a template parameter. Placeholders use a character normally not found in D code: \DD{\#}. The $n^{th}$ parameter is \DD{\#n}, starting from 0. As a concession to practicality, a lone \DD{\#} is considered equivalent to \DD{\#0}. Args to be put into the string are passed as standard (non-template) parameters and can be of any type.

\begin{dcode}
template interpolate(string code)
{
    string interpolate(Args...)(Args args) {
        string[] stringified;
        foreach(index, arg; args) stringified ~= to!string(arg);

        string result;
        int i;
        int zero = to!int('0');

        while (i < code.length) {
            if (code[i] == '#') {
                int j = 1;
                int index;
                while (i+j < code.length
                    && to!int(code[i+j])-zero >= 0
                    && to!int(code[i+j])-zero <= 9)
                {
                    index = index*10 + to!int(code[i+j])-zero;
                    ++j;
                }

                result ~= stringified[index];
                i += j;
            }
            else {
                result ~= code[i];
                ++i;
            }
        }

        return result;
    }
}
\end{dcode}

\TODO{The syntax could be extended somewhat: inserting multiple strings, inserting a range of strings, all arguments to the end.}

\subsection{Example: extending \DD{std.functional.binaryFun}}\label{naryfun}

\unfinished{This one is dear to my heart. Mapping $n$ ranges in parallel is one of the first things that I wanted to do with ranges, for examples to create ranges of structs with constructor taking more than one parameter.}

Phobos has two really interesting templates: \stdanchor{functional}{unaryFun} and \stdanchor{functional}{binaryFun}.

\TODO{Explain that this aims to extend that to n-args functions.}


\begin{dcode}
bool isaz(char c) {
    return c >= 'a' && c <= 'z';
}

bool isAZ(char c) {
    return c >= 'A' && c <= 'Z';
}

bool isNotLetter(char c) {
    return !isaz(c) && !isAZ(c);
}

int letternum(char c) {
    return to!int(c) - to!int('a') + 1;
}

int arity(string s) {
    if (s.length == 0) return 0;

    int arity;
    string padded = " " ~ s ~ " ";
    foreach(i, c; padded[0..$-2])
        if (isaz(padded[i+1]) 
         && isNotLetter(padded[i]) 
         && isNotLetter(padded[i+2]))
            arity = letternum(padded[i+1]) > arity ? 
                    letternum(padded[i+1]) 
                  : arity;
    return arity;
}

string templateTypes(int arit) {
    if (arit == 0) return "";
    if (arit == 1) return "A";

    string result;
    foreach(i; 0..arit)
        result ~= "ABCDEFGHIJKLMNOPQRSTUVWXYZ"[i] ~ ", ";

    return result[0..$-2];
}

string params(int arit) {
    if (arit == 0) return "";
    if (arit == 1) return "A a";

    string result;
    foreach(i; 0..arit)
        result ~= "ABCDEFGHIJKLMNOPQRSTUVWXYZ"[i]
               ~ " " ~ "abcdefghijklmnopqrstuvwxyz"[i]
               ~ ", ";

    return result[0..$-2];
}

string naryFunBody(string code, int arit) {
    return interpolate!"auto ref naryFun(#0)(#1) { return #2;}"
                      (templateTypes(arit), params(arit), code);
}

template naryFun(string code, int arit = arity(code))
{
    mixin(naryFunBody(code, arit));
}
\end{dcode}

\section{\D{\_\_traits}}\label{traits}

\unfinished{I should maybe add an example about how to get all parents from a class type. Also, iterating on all members of a type, or extracting types and names from an aggregate (a module, a class or a struct).}

The general \D{\_\_traits} syntax can be found online \href{www.dlang.org/traits.html}{here}. Traits are basically another compile-time introspection\index{compile-time!introspection!with \_\_traits@with \D{\_\_traits}}\index{introspection!with \_\_traits@with \D{\_\_traits}} tool, complementary to the \D{is} expression (see  \autoref{isexpression}). Most of time, \D{\_\_traits} will return \D{true} or \D{false} for simple type-introspection questions (is this type or symbol an abstract class, or a final function?). As D is wont to do, these questions are sometimes ones you could ask using \D{is} or template constraints, but sometimes not. What's interesting is that you can do some introspection on types, but also on symbols or expressions.

\subsection{Yes/No Questions with \D{\_\_traits}}\label{yesnoquestionsontypes}

Seeing how this is a document on templates and that we have already seen many introspection tools, here is a quick list of what yes/no questions you can ask which can or \emph{cannot} be tested with \D{is}:\footnote{As with any other affirmation in this document, readers should feel free to prove me wrong. That shouldn't be too difficult.}

\begin{table}[htb]
\centering
\begin{tabular}[c]{|c|c|}
\hline
Question & Doable with other tools? \\
\hline
\hline
isArithmetic & Yes \\
isAssociativeArray & Yes \\
isFloating & Yes \\
isIntegral & Yes \\
isScalar & Yes \\
isStaticArray & Yes \\
isUnsigned & Yes \\
\hline
\hline
isAbstractClass & No \\
isFinalClass & No \\
isVirtualFunction & No \\
isAbstractFunction & No \\
isFinalFunction & No \\
isStaticFunction & No \\
\hline
\hline
isRef & No \\
isOut & No \\
isLazy & No \\
\hline
\hline
hasMember & No (Yes?) \\
isSame & No \\
compiles & Yes (in a way) \\
\hline
\end{tabular}
\caption{Comparison between \D{\_\_traits} and other introspection tools}
\label{table:traits}
\end{table}

These can all be quite useful in your code, but I'll shy away from them since they are not heavily template-related. More interesting in my opinion is using \D{\_\_traits} to get new information about a type. These are really different from other introspection tools and I will deal with them in more detail right now.

\subsection{\DD{identifier}}

\DD{identifier} gives you a symbol's name as a string. This is quite interesting, since some symbols are what I'd call \emph{active}: for example, if \DD{foo} is a function, \DD{foo.stringof} will try to first call \DD{foo} and the \DD{.stringof} will fail. Also, \DD{.stringof}, though eminently useful, sometimes returns strangely formatted strings. \DD{identifier} is much more well-behaved.

Let's get back to one of the very first template in this doc, \DD{NameOf} on page \pageref{templatedeclarationexamples}. Initially, it was coded like this:

\begin{dcode}
template NameOf(alias a)
{
    enum string name = a.stringof; // enum: manifest constant
                                   // determined at compile-time
}
\end{dcode}

But, this fail for functions:

\begin{dcode}
int foo(int i, int j) { return i+j;}

auto name = NameOf!foo; // Error, 'foo' must be called with 2 arguments
\end{dcode}

It's much better to use \D{\_\_traits} (also, the eponymous trick (\ref{eponymous}):

\begin{dcode}
template nameOf(alias a)
{
    enum string nameOf = __traits(identifier, a);
}

int foo(int i, int j) { return i+j;}

enum name = nameOf!foo; // name is available at compile-time

assert(name == "foo");
\end{dcode}

Note that this works for many (all?) kinds of symbols: template names, class names, even modules:
auto
\begin{dcode}
import std.typecons;

enum name2 = nameOf!(nameOf); // "nameOf"
enum name3 = nameOf!(std.typecons); //"typecons"
\end{dcode}

\subsection{\DD{getMember}}

In a nutshell, \D{\_\_traits}\DD{(getMember, name, "member")} will give you direct access to \DD{name.member}. This is the real member: you can get its value (if any), set it anew, etc. Any D construct with members is OK as a \DD{name}. If you wonder why the aggregate is called directly by its own name whereas the member needs a string, it's because the aggregate is a valid symbol (it exists by itself), when the member's name has no existence outside the aggregate (or even worse, may refer to another, unrelated, construct).

\aparte{Aggregates}{I'll use \emph{aggregate} as a catch-all term for any D construct that has members. Structs and classes are obvious aggregates, but it's interesting to keep in mind that templates too have members (remember section \ref{instantiating}? A template is a named, parameterized scope). So all \D{\_\_traits} calls shown in this section can be used on templates. That's interesting to keep in mind. Even more interesting, to my eyes, is that \emph{modules} are also aggregates even though they are not first-class citizens in D-land. You'll see examples of this in the following sections.}

\subsection{\DD{allMembers}}
\index{\_\_traits!allMembers@\D{\_\_traits}!\DD{allMembers}}

This one is cool. Given an aggregate name, \D{\_\_traits}\DD{(allMembers, aggregate)} will return a tuple of string literals, each of which is the name of a member. For a class, the parent classes' members are also included. Built-in properties (like \DD{.sizeof}) are not included in that list. Note that I did say 'tuple': it's a bit more powerful than an array, because it can be iterated over at compile-time.

The names are not repeated for overloads, but see the next section for a way to get the overloads.

\begin{dcode}
class MyClass 
{
    int i;
    
    this() { i = 0;}
    this(int j) { i = j;}
    ~this() { }
    
    void foo() { ++i;}
    int foo(int j) { return i+j;}
}

// Put in an array for a more human-readable printing
enum myMembers = [__traits(allMembers, MyClass)];

// See "i" and "foo" in the middle of standard class members
// "foo" appears only once, despite it being overloaded.
assert(myMembers == ["i", "__ctor", "__dtor", "foo", "toString", 
                     "toHash", "opCmp", "opEquals", "Monitor", "factory"]);
\end{dcode}

So the above code is a nice way to get members, both fields (like \DD{i}) and methods (like \DD{foo}). In case you wonder what \DD{"\_\_ctor"} and \DD{"\_\_dtor"} are, it's the internal D name for constructors and destructors. But it's perfectly usable in your code! For structs, the list far less cluttered since they only get the constructor and destructor's names and \DD{opAssign}, the assignment operator (\DD{=}).

Since this trait returns strings, it can be plugged directly into \DD{getMember}. See a bit farther down a way to get a nice list of all members and their types.

Now, let's ramp things up a bit: what about class and struct templates? Let's try this out:

\begin{dcode}
class MyClass(T)
{
    T field;
}

assert([__traits(allMembers, MyClass)] == ["MyClass"]);
\end{dcode}

Oh, what happened? Remember from \autoref{basics} that struct, classes and functions templates are just syntactic sugar for the 'real' syntax:

\begin{dcode}
template MyClass(T)
{
    class MyClass
    {
        T field;
    }
}
\end{dcode}

So, \DD{MyClass} is just the name of the external template, whose one and only member is\ldots the \DD{MyClass} class. So all is well. If you instantiate the template, it functions as you might expect:

\begin{dcode}
assert([__traits(allMembers, MyClass!int)] 
    == ["field", "toString", "toHash", "opCmp", 
        "opEquals", "Monitor", "factory"]);
\end{dcode}

If you remember one of the very first use we saw for templates in \ref{instantiating}, that is as a named, parameterize scope,\index{scope} this can give rise to interesting introspection:\index{introspection}

\begin{dcode}
template Temp(A, B)
{
    A a;
    B foo(A a, B b) { return b;}
    int i;
    alias A    AType;
    alias A[B] AAType;
}

assert([__traits(allMembers, Temp)] 
    == ["a", "foo", "i", "AType", "AAType"]);
assert([__traits(allMembers, Temp!(double,string))] 
    == ["a","foo", "i", "AType", "AAType"]);
\end{dcode} 

As you can see, you also get aliases' names. By the way, this is true for structs and templates also\ldots

Another fun fact is that D modules\index{modules!as aggregates} are amenable to \D{\_\_traits}'s calls:

\begin{dcode}
// file a.d
module a;

int foo(int i) { return i;}
class MyClass { }
\end{dcode}
\begin{dcode}
// file b.d
module b;

import a; // bring 'a' as a symbol in the module's scope.
import std.algorithm;

enum aMembers = [__traits(allMembers, a)];
enum algos = [__traits(allMembers, std.algorithm)]; // huuuge list of names

enum myself = [__traits(allMembers, b)]; // auto-introspection!

void main()
{
    assert(aMembers == ["object", "foo", "MyClass"]);
    assert(myself   == ["object", "a", "std", "main"]);
}
\end{dcode}

In the previous code, you see that among \DD{a} members, there is \DD{"object"} (the implicit \DD{object.d} module\index{module!implicit object.d module@module!implicit \DD{object.d} module} imported by the runtime). And since auto-introspection is possible, we also get \DD{b}'s members. There, there is also \DD{"a"}, as it's imported into \DD{b}, and \DD{"std"}, the global package that shows here whenever you import a \DD{std.*} module. It would be easy to imagine a template that recursively explore the members, find the modules and try to recurse into them to get a complete import-tree with a template. Halas \DD{"std"} blocks that, since the package itself do not have a member.\footnote{ If someone finds a way to determine with a template that \DD{b} imports \DD{std.algorithm}, I'd be delighted to see how it's done!}

As for the other aggregates, you get the aliases also, and the unit tests defined in the module (for those of you curious about it, they are named \DD{\_\_unittestXXX} where \DD{XXX} is a number. Their type is more or less \D{void delegate}\DD{()}).\footnote{ I didn't try static constructors in modules. Don't hesitate to play with them and tell me what happens.}

What's the point of inspecting a module, you say? Well, first that was just for fun and to see if I could duplicate a module or create a struct with an equivalent members list (all forwarding to the module's own members). But the real deal for me was when using string mixins to generate some type. If the user uses the mixin in its own module, it could create conflicts with already-existing names. So I searched for a way for a mixin template to inspect the module it's currently being instantiated in. 

\subsection{\DD{derivedMembers}}

Really it's the same as above, except that you do not get a class' parents members, only the class' own members.

\subsection{\DD{getOverloads}}
\index{\_\_traits!getOverload@\D{\_\_traits}!\DD{getOverload}}

Given:

\begin{itemize}
\item an aggregate name or an instance of that agregrate
\item a member name as a string
\end{itemize}

Then, \D{\_\_traits}\DD{(getOverloads, name, "member")} will gives you a tuple of all local overloads of \DD{name.member}. By 'local', I mean that for classes, you do not get the parents classes overloads. There is a difference between using \DD{getOverloads} on a type and on an instance: in the former case, you get \ldots, well I don't know exactly what it is you get but it's a bit akin to static members. On an instance, you get an expression that is the exact member overload. The \href{http://dlang.org/traits.html#getOverloads}{documentation} is misleading: it says \DD{getOverloads} returns an array, but than cannot be the case, as all overloads must have a different type. What's cool is that, by using \DD{"\_\_ctor"}, you also get a direct access to a type's constructor overloads. That can be quite handy in some cases.



\subsection{Getting All Members, Even Overloaded Ones}
\index{\_\_traits!getting all members@\D{\_\_traits}!getting all members}

Now, if you're like me, the urge to mix \DD{allMembers} (which returns the members' names without overloads) and \DD{getOverloads}  (which returns the overload of \emph{one} member) is quite great. So let's do that.

First, a bit of machinery: I want the members to be described by a name and a type. Let's create a struct holder, templated of course:

\TODO{Getting qualified names would be better.}

\begin{dcode}
struct Member(string n, T)
{
    enum name = n; // for external access
    alias T Type;
}
\end{dcode}

Given a member name, we'll need a template that delivers the associated \DD{Member}:

\begin{dcode}
template MakeMember(alias member)
{
    alias Member!(nameOf!member, typeof(member)) MakeMember;
}
\end{dcode}

Now, given an aggregate name and a member name (as a string, since these do not exist by themselves), we want a list of \DD{Member} structs holding all the information:

\begin{dcode}
emplate Overloads(alias a, string member)
{
    alias OverloadsImpl!(__traits(getOverloads,a, member)).result Overloads;
}

template OverloadsImpl(A...)
{
    alias staticMap!(MakeNameAndType,A) result;
}
\end{dcode}

It's a two-stage template as explained in sections \ref{eponymous} (eponymous templates) and \ref{templatesintemplates} (to get the overloads expansion as a tuple parameter). \DD{staticMap} is explained in section \ref{staticMap}.

Now, that already works:

\begin{dcode}
class MyClass 
{
    int i;
    
    this() { i = 0;}
    this(int j) { i = j;}
    ~this() { }
    
    void foo(int j) { ++i;}
    int foo(int j, int k = 0) { return i+j;}
}

void main()
{
    alias Overloads!(MyClass, "foo") o;
    
    /* 
    prints:
    foo, of type: void(int j)
    foo, of type: int(int j, int k = 0)
    */
    foreach(elem; o) 
        writeln(elem.name, ", of type: ", elem.Type.stringof);
}
\end{dcode}

We indeed got two \DD{Member} instances, one for each overload. Each \DD{Member} holds the name \DD{"foo"} and the overload's type. Except, there is a catch: for a field, there are no overloads. We need to pay attention to that in \DD{Overloads}:

\begin{dcode}
template Overloads(alias a, string member)
{
    static if (__traits(compiles, __traits(getOverloads, a, member)))
        alias OverloadsImpl!(__traits(getOverloads,a, member)).result
              Overloads;
    else
        alias TypeTuple!(Member!(member, typeof(__traits(getMember, a, member)))) 
              Overloads;
}
\end{dcode}

Line 3 is an example of the \emph{very handy} \D{\_\_traits}\DD{(compiles, someExpression)} construct alluded to in section \ref{yesnoquestionsontypes}. Line 7 can be a bit daunting, but it just creates the one-element tuple \D{\_\_traits} should have returned.

The last step is to get this for all members of a given aggregate. It's quite easy:

\begin{dcode}
template GetOverloads(alias a)
{
    template GetOverloads(string member)
    {
        alias Overloads!(a, member) GetOverloads;
    }
}

template AllMembers(alias a)
{
    alias staticMap!(GetOverloads!(a), __traits(allMembers, a)) AllMembers; 
}
\end{dcode}

The strange \DD{GetOverloads} two-stage construction is just a way to map it more easily on line 11. So, this was quite long to explain, but it works nicely:

\begin{dcode}
void main()
{
    alias AllMembers!MyClass o;
    /*
    prints:
    i, of type: int
    __ctor, of type: MyClass()
    __ctor, of type: MyClass(int j)
    __dtor, of type: void()
    foo, of type: void(int j)
    foo, of type: int(int j, int k = 0)
    toString, of type: string()
    toHash, of type: uint()
    opCmp, of type: int(Object o)
    opEquals, of type: bool(Object o)
    opEquals, of type: bool(Object lhs, Object rhs)
    factory, of type: Object(string classname)
    */
    foreach(elem; o) 
        writeln(elem.name, ", of type: ", elem.Type.stringof);    
}
\end{dcode}

That's cool, every member and overload is acounted for: the two contructors are there, the destructor also, and of course \DD{i}, of type \D{int}.

\TODO{Example: store all members in a hashtable or a polymorphic association list. As a mixin, to be put inside types to enable runtime reflection? (a.send("someMethod", args), a.setInstanceVariable("i",5))}

\subsection{\DD{getVirtualFunctions}}

It's in the same family than \DD{getOverloads} and such. It'll give you the list of virtual overloads for a class method. Given a class name, finding all overloads of all fields, even overriden ones, is let as a exercise to the reader.

\subsection{\DD{parent}}\index{\_\_traits!parent@\D{\_\_traits}!\DD{parent}}\index{qualified name}

\D{\_\_traits}\DD{(parents, symbol)} will return the symbol that's the parent of it. It's \emph{not} the parent in a 'class hierarchy' sense,\footnote{ If you came here to see a way to get all parents of a class, see section \ref{classhierarchy}.} it deals with qualified names and strip one level. Once you reach the toplevel scope, it returns the module name (this can be dangerous, because modules themselves do not have parents). See:

\begin{dcode}
module test;

class C
{
    int i;
    int foo(int j)
    {
        int k; // k is "test.C.foo.k"
        assert(nameOf!(__traits(parent, k)) == "foo");
        return i+k;
    }
}

// toplevel (C is in fact test.C)
static assert(nameOf!(__traits(parent, C)) == "test");

void main() 
{
    auto c = new C(); // c is "test.main.c"
    assert(nameOf!(__traits(parent, c)) == "main");
    assert(nameOf!(__traits(parent, c.i)) == "C");
    c.foo(1);
}
\end{dcode}

Even if there is no \DD{qualifiedIdentifier} in \D{\_\_traits}, we can construct a template to get it:

\begin{dcode}
template qualifiedName(alias a)
{
    // does it have a parent?
    static if (__traits(compiles, __traits(parent, a)))
    // If yes, get the name and recurse
        enum qualifiedName = qualifiedName!(__traits(parent, a)) 
                           ~  "." ~ nameOf!(a);
    // if not, it's a module name. Stop there.
    else
        enum qualifiedName = nameOf!a;
}

/* given the previous C class */

void main()
{
    auto c = new C();
    assert(qualifiedName!c == "test.main.c");
    assert(qualifiedName!(c.foo) == "test.C.foo"); // same in both cases
    assert(qualifiedName!(C.foo) == "test.C.foo");
}
\end{dcode}

\subsection{Local Scope Name}
\index{scope!local scope name}

Sometimes, when dealing with mixin templates (\ref{mixintemplates}) or string mixins (\ref{stringmixins}), you'll inject code in an unknown scope. To get your way round, it can be useful to get the local scope's name. Intuitively, the previous example could help with that: just create a local variable, get the qualified name of its parent to determine in which scope the mixin is. Then, expose the scope name. Let's call this one \DD{scopeName} and the associated inquisitive template \DD{getScopeName}.

\begin{dcode}
mixin template getScopeName()
{
    enum scopeName = qualifiedName!(\_\_traits(parent, scopeName));
}
\end{dcode}

The idea is to declare local enum called \DD{scopeName} and take the qualified name of its own parent in the same expression (yes, that works!).

To use \DD{getScopeName}, just mix it in where you need a local scope name:

\begin{dcode}
module test;

class C
{
    mixin getScopeName; // 1
    
    int i;
    int foo(int j)
    {
        int k;
        mixin getScopeName; // 2
        writeln(scopeName);
        return i+k;
    }    
}

void main()
{
    auto c = new C();
    writeln(c.scopeName); // "test.C" (1)
    c.foo(1);             // "test.C.foo" (2)
    
    mixin getScopeName; // 3
    writeln(scopeName);   // "test.main" (3)
}
\end{dcode}

\section{Wrapping it all Together}\label{wrappingitup}

\unfinished{I have to dig some cute examples in my code base. Using string mixins and CTFE and templates all together.}
