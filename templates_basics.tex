\newpage
\part{Basics}\label{basics}

A template is a recipe, a blueprint that will generate code at your command and according to compile-time parameters you'll give. Templates are \emph{parameterized code}. Each template definition is written once in a module and can then be instantiated many times with different parameters, possibly resulting in quite different code, depending on the arguments you used.

\section{Template Declarations}\label{declarations}

Here is the syntax for a template declaration:

\index{template!declaration}
\index{syntax!template declaration}\begin{dcode}
template templateName(list, of, parameters)
{
  // Some syntactically correct declarations here
  // The arguments are accessible inside the template scope.
}
\end{dcode}


\DD{templateName} is your usual D identifier and the list of parameters is a comma-separated list of zero or more template parameters. These can be:

\begin{description}

\item[Types:]\index{template!parameters!type} 

An \DD{identifier} alone by itself is considered a type name. The common D style is to use identifiers beginning with a capital letter (\DD{Range}, \DD{Rest}), as for any user-defined types. Many D templates use the C++\index{C++} tradition of one-capital-letter names for types, starting from \DD{T} (\DD{U}, \DD{V}, \ldots). Do not feel constrained by this, use what makes your templates most easy to understand.

\item[Aliases: ]\index{template!parameters!alias} 

These will capture symbols: variable names, class names, even other template names. They will also accept many compile-time literals: strings, arrays, function literals, \ldots Mostly, if you need a widely-accepting template, use an alias parameter. They will \emph{not} accept built-in types as arguments, however. You declare them with \D{alias} \DD{identifier}.

\item[Literal values:]\index{template!parameters!literal value} 

They can be integral values (\D{int}, \D{ulong}, \ldots), enum-based, strings, chars, floating-point values or boolean values. I think the rule is that any expression that can be evaluated at compile-time is OK. They are all declared like this: \D{typeName} \DD{iden\-ti\-fier}. For example : \D{int} \DD{depth} or \D{string} \DD{name}.

\item[Template parameters tuples:]\index{template!parameters!tuple} 

Template parameters tuples will capture un\-der one id\-en\-ti\-fier an en\-ti\-re list of tem\-plate pa\-ra\-me\-ters (types, names, literals, \ldots). Tuples will store any template argument you will throw at them. If no argument is passed, you will just get a zero-length tuple. Really, as they can deal with types as well as symbols, these tuples are a bit of a mongrel type but they are wonderfully powerful and easy to use, as you will see in section \ref{tuples}. The syntax is \DD{identifier...} (yes, three dots) and tuples must be the last parameter of a template. 
\end{description}

Of those, types and aliases are the most common, while floating point values are fairly rare (their use as arguments for compile-time calculations have been superseded by D's Compile-Time Function Evaluation, aka CTFE\index{Compile-Time Function Evaluation|see{CTFE}}\index{CTFE}, described in section \ref{ctfe}). You'll see different uses of these parameters in this document.

Note that pointers, arrays, objects (instantiated classes), structs or functions are not part of this list. But as I said, alias parameters\index{template!parameters!alias} allow you to capture and use array, class, function or struct \emph{names} and then access their capacities.

\aparte{Aliases, Symbols and Names}{There is big difference between built-in types (like \D{int} or \D{double}\DD{[3]}) and user-defined types. A user-defined type, say a class called \DD{MyClass}, is a type name. So, it's \emph{both} a type (the class \DD{MyClass}, accepted by type templates arguments) and a name (\DD{MyClass}, accepted by \D{alias} template parameters). On the other hand, \D{int}, being a D keyword is not a symbol nor a name. It's just a type. You cannot pass it to an alias template parameter.}
The template body can contain any standard D declarations: variable, function, class, interface, other templates, alias declarations,\ldots The only exception I can think of is declaring a \D{module}\index{module}, as this is done at the top-level scope\index{scope!top-level}. 

\aparte{Syntax and Semantics} {And then there is a catch: code inside a \D{template} declaration must only by syntactically correct D code (that is: code that looks like D code). The semantics is checked only during instantiation. That means you can code happily, writing templates upon templates and the compiler won't bat an eye if you do not exercise your templates by instantiating them.}

Inside the template body, the parameters are all accessible as placeholders for the future arguments. Also, the template's own name refers to its current instantiation when the code is generated. This is mostly used in struct (see section \ref{structtemplates}) and class (\ref{classtemplates}) templates.

Here are some template declaration examples:\index{example!template definition}

\begin{dcode}
template ArrayOf(T) // T is a type
{
    alias T[] ArrayType;
    alias T ElementType;
    immutable T t; // storage class

}

template Transformer(From, To) // From and To are types, too
{
    To transform(From from) { /* some code that returns a To*/}

    class Modificator
    {
        From f;
        To t;
        this(From f) { ... }
    }
}

template NameOf(alias a)
{
    enum string name = a.stringof; // enum: manifest constant
                                   // determined at compile-time
}

template ComplicatedOne(T, string s, alias a, bool b, int i)
{ /* some code using T, s, a, b and i */ }

template Minimalist() {} // Zero-parameter template declaration.

template OneOrMore(FirstType, Rest...) // Rest is a tuple.
{ ... }

template ZeroOrMore(Types...) // Types is a tuple.
{ ... }

template Multiple(T)     { ... } // One arg version.
template Multiple(T,U)   { ... } // Two args,
template Multiple(T,U,V) { ... } // and three.
\end{dcode}

The real syntax for template declarations is slightly more complex, I'll introduce more of it in the next sections (You'll see for example type restrictions in section \ref{specializations}, default values in section \ref{default}, instantiation constraints in \ref{constraints} and more on tuples in section \ref{tuples}).

There is a limitation that's interesting to keep in mind: templates can be declared in almost any scope\index{scope}, except inside a (standard) function.

\aparte{\D{enum}}{In the previous code, see line 22? It defines a \D{string} called \DD{name} as a member of \DD{NameOf}. The \D{enum} placed right before means \DD{name} is a compile-time constant.\index{compile-time!constant (\D{enum})} You can see it as a kind of storage class,\index{storage class!enum as a kind of@\D{enum} as a kind of} in the line of \D{immutable} or \D{const}, one that means the value is totally defined and fixed at runtime. You'll see numerous examples of \D{enum} in this document.}

\section{Instantiating a Template}\label{instantiating}

To instantiate a template, use the following syntax:

\index{syntax!template instantiation}
\begin{dcode}
templateName!(list, of, arguments)
\end{dcode}


Note the `\DD{!}' before the comma-separated argument list. If the argument list contains only one argument (one token), you can drop the parenthesis:

\index{syntax!template instantiation}
\begin{dcode}
templateName!argument
\end{dcode}



\aparte{Templates as templates arguments\index{template!arguments!templates as}}{Arguments can themselves be the result of another template instantiation. If a template returns a type upon instantiation, it's perfectly OK to use it inside another template argument list. In this document you'll regularly see Matrioshka calls like this: \DD{firstTemp!(secondTempl!(Arguments), OtherArguments).}}


The compiler will have a look at the declarations (if more than one template were declared with the called name) and select the one with the correct number of arguments and the correct types to instantiate. If more than one template can be instantiated, it will complain  and stop there (though, have a look on template specializations in section \ref{specializations} and template constraints in section \ref{constraints}).

When you instantiate a template, the global effect is that a new named scope\index{scope!named} is created in the template declaration scope\index{scope!template declaration}. The name of this new scope is the template name with its argument list: \DD{templateName!(args)}. Inside the scope, the parameters are now 'replaced' with the corresponding arguments (storage classes get applied, variables are initialized, \ldots). Here's what possible instantiations of the templates declared in section \ref{declarations} might look like:
\index{example!template instantiation}
\begin{dcode}
ArrayOf!int

Transformer!(double,int) // From is an alias for the type double
                         // To for the type int

struct MyStruct { ... }
NameOf!(MyStruct) // "MyStruct" is a identifier -> captured by alias

ComplicatedOne!( int[]   // a type
               , "Hello" // a string literal
               , ArrayOf // a name (here the ArrayOf template)
               , true    // a boolean literal
               , 1+2     // calculated to be the integral '3'.
               )

Minimalist!()
// or even:
Minimalist

OneOrMore!( int                   // FirstType is int.
          , double, string, "abc" // Rest is (double,string,"abc")
          )

ZeroOrMore!(int)               // Types is a 1-element tuple: (int)
ZeroOrMore!(int,double,string) // Types is (int,double,string)
ZeroOrMore!()                  // Types is the empty tuple: ()

Multiple!(int)               // Selects the one-arg version
Multiple!(int,double,string) // The three args version.
Multiple!()                  // Error! No 0-arg version
\end{dcode}

Outside the scope\index{scope!template instantiation} (that is, where you put the template instantiation in your own code), the internal declarations are accessible by fully qualifying them:

\begin{dcode}
// ArrayType is accessible (it's int[])
// array is a completly standard dynamic array of ints.
ArrayOf!(int).ArrayType array; 
ArrayOf!(int).ElementType element; // the same, element is an int.

// the transform function is accessible. Instantiated like this, 
// it's a function from string to double. 
auto d = Transformer!(string,double).transform("abc"); // d is a double
\end{dcode}

Obviously, using templates like this, with their full name, is a pain. The nifty D \D{alias} declaration is your friend:

\begin{dcode}
alias Transformer!(string,double) StoD;

auto d = StoD.transform("abc");
auto m = new StoD.Modificator("abc"); // StoD.Modificator is a class
                                      // storing a string and a double.
\end{dcode}

You must keep in mind that instantiating a template means generating code. Using different arguments at different places in your code will instantiate \emph{as many differently named scopes}\index{scope!multiple templates instantiations}. This is a major difference with \emph{generics}\index{generics} in languages like Java\index{Java} or C\#\index{C\#}, where generic code is created only once and type erasure is used to link all this together. On the other hand, trying to instantiate many times the `same' template (ie: with the same arguments) will only create one piece of code.

\begin{dcode}
alias Transformer!(string,double) StoD;
alias Transformer!(double,string) DtoS;
alias Transformer!(string,int)    StoI;
// Now we can use three different functions and three different classes.
\end{dcode}

\aparte{\D{void}}{Note that \D{void}\index{void@\D{void}!as a template parameter}\index{template!parameters!void@\D{void}} is a D type and, as such, a possible template argument for a type parameter. Just be careful because many templates make no sense when \D{void} is used as a type. In the following sections and in the appendix, you'll see ways to restrict arguments to some types.}

\section{Templates Building Blocks}\label{buildingblocks}

Up to now, templates must seem not that interesting to you, even with a simple declaration and instantiation syntax. But wait! D introduced a few nifty tricks that both simplify and greatly expand templates use. This section will introduce you to your future best friends, the foundations on which your templates will be built.

\subsection{The Eponymous Trick}\label{eponymous}
\index{eponymous trick}

If a template declares only one symbol with the same name (greek: \emph{epo-nymous}) as the enclosing template, that symbol is assumed to be refered to when the template is instantiated. This one is pretty good to clean your code:
\index{example!eponymous trick}
\begin{dcode}
template pair(T)
{
    T[] pair(T t) { return [t,t];} // declares only pair
}

auto array = pair!(int)(1);  // no need to do pair!(int).pair(1)
\end{dcode}

\begin{dcode}
template NameOf(alias name)
{
    enum string NameOf = name.stringof;
}

int foo(int i) { return i+1;}
auto s1 = NameOf!(foo); // s1 is "foo"
auto s2 = NameOf!(NameOf)
\end{dcode}

A limitation is that the eponymous trick works \emph{only} if you define one (and only one) symbol. Even if the other symbols are private and such, they will break the eponymous substitution. This may change at a latter time, as the D developers have shown interest for changing this, but for the time being, you cannot do:

\begin{dcode}
template ManyAlias(T, U)
    T[] firstArray;
    U[] secondArray;
    T[U] ManyAlias; // Halas, ET substitution doesn't work here
}

// Hoping for ManyAlias!(int,string).ManyAlias
ManyAlias!(int, string) = ["abc":0, "def":1]; 

ManyAlias!(int, string).firstArray = [0,1,2,3];
\end{dcode}

But, will you ask, what if I want client code to be clean and readable by using the eponymous trick, when at the same time I need to internally create many symbols in my template? In that case, create a secondary template with as many symbols as you need, and expose its final result through a (for example) \DD{result} symbol. The first template can instantiate the second one, and refer only to the \DD{.result} name:
\index{example!secondary template}
\begin{dcode}
/* your code */

// primary template
template MyTemplate(T, U, V)
{
    // eponymous trick activated!     
    enum MyTemplate = MyTemplateImpl!(T,U,V).result;
}

// secondary (hidden) template
template MyTemplateImpl(T, U, V)
{
    // The real work is done here 
    // Use as many symbols as you need.
    alias symbol1 ...
    enum otherName = ...
    enum result = ...
}
\end{dcode}

\begin{dcode}
/* client code */
MyTemplate!(int,string,double[]) someValue;
(...)
\end{dcode}

You can find an example of this two-templates idiom\index{idiom!two templates} in Phobos, for example in \stdanchor{functional}{unaryFun} or \stdanchor{functional}{binaryFun}.

\subsection{Inner \D{alias}}\label{inneralias}

A common use for templates is to do some type magics: deducing types, assembling them in new way, etc. Types are not first-class entities in D (there is no `\D{type}' type), but they can easily be manipulated as any other symbol, by aliasing them. So, when a template has to expose a type, it's done by aliasing it to a new name.
\index{example!exposing through an alias}
\begin{dcode}
template AllArraysOf(T)
{
    alias T    Element;
    alias T*   PointerTo;
    alias T[]  DynamicArray;
    alias T[1] StaticArray;
    alias T[T] AssociativeArray;
}
\end{dcode}

\aparte{Exposing Template Parameters}{Though they are part of a template's name, its parameters are \emph{not} directly accessible externally. Keep in mind that a template name is just a scope name\index{scope}. Once it's instantiated, all the \DD{T}s and \DD{U}s and such do not exist anymore. If you need them externally, expose them through a template member, as is done with \DD{AllArraysOf.Element}. You will find other examples of this in section \ref{structtemplates} on structs templates and section \ref{classtemplates} on class templates.}

\subsection{\D{static if}}\label{staticif}

\subsubsection{Syntax}

The \D{static if} construct\footnote{ It's \emph{both} an expression and a declaration, so I'll call it a construct.}
let you decide between two code paths at compile time. It's not specific to templates (you can use it in other part of your code), but it's incredibly useful to have your templates adapt themselves to the arguments. That way, using compile-time-calculated predicates based on the template arguments, you'll generate different code and customize the template to your need.

The syntax is:
\index{syntax!static if}
\begin{dcode}
static if (compileTimeExpression)
{
     /* Code created if compileTimeExpression is evaluated to true */
}
else /* optional */
{
     /* Code created if it's false */
}
\end{dcode}

Something really important here is a bit of compiler magic: once the code path is selected, the resulting code is instantiated in the template body, but without the curly braces. Otherwise that would create a local scope\index{scope!static if@\D{static if}}, hiding what's happening inside to affect the outside and would drastically limit the interest of \D{static if}. So the curly braces are only there to group the statements together. 

If there is only one statement, you can get rid of the braces entirely, something you'll see frequently in D code. For example, suppose you need a template that `returns' \D{true} if the passed type is a dynamic array and \D{false} otherwise (this kind of predicate template\index{template!predicate templates} is developed a bit more in section \ref{predicates}).

\begin{dcode}
template isDynamicArray(T)
{
    static if (is(T t == U[], U))
        enum isDynamicArray = true;
    else
        enum isDynamicArray = false;
}
\end{dcode}

As you can see, no curly braces after \D{static if} and we are also using the eponymous trick\index{eponymous trick} (\DD{isDynamicArray} is the only symbol defined by the template and its type is automatically deduced by the compiler), resulting in a very clean syntax. The \D{is}\DD{()}\index{is expression@\D{is} expression} part is a way to get compile-time introspection\index{compile-time!introspection} which goes hand in hand with \D{static if}. There is a crash course on it at the end of this document (see Appendix \ref{isexpression}).

\subsubsection{Optional Code}

A common use of \D{static if} is to enable or disable code: a single \D{static if} without an \D{else} clause will generate code only when the condition is true. You can find many examples of this idiom\index{idiom!enabling/disabling code} in \std{range}\index{range!higher-level ranges} where higher-level ranges (ranges wrapping other ranges) will activate some functionality only if the wrapped range can support it, like this:\index{example!enabling/disabling code}


\begin{dcode}
/* We are inside a MyRange templated struct, wrapping an R. */

    R innerRange;

/* some code that exist in all instantiations of MyRange */
(...)

/* optional code */
static if (hasLength!R) // does innerRange has a .length() method?
    auto length()       // then MyRange has one also
    {
        return innerRange.length;
    }

static if (isInfinite!R)     // Is innerRange an infinite range?
    enum bool empty = false; // Then MyRange is also infinite.
// And so on...
\end{dcode}

\subsubsection{Nested \D{static if}s}

\D{static if}s can be nested: just put another \D{static if} after \D{else}. Here is a template selecting an alias:

\index{syntax!nested \D{static if}s}
\index{example!nested \D{static if}s}
\index{static if@\D{static if}!nested}
\begin{dcode}
import std.traits: isIntegral, isFloatingPoint;

template selector(T, alias intFoo, alias floatFoo, alias defaultFoo)
{
    static if (isIntegral!T)
       alias intFoo selector;
    else static if (isFloatingPoint!T)
       alias floatFoo selector; 
    else // default case
        alias defaultFoo selector;
}
\end{dcode}

If you need a sort of \D{static switch} construct, see section \ref{examples:staticswitch}.

\subsubsection{Recursion with \D{static if}}\label{staticifrecursion}

\paragraph{Rank:}\label{rank}

Now, let's use \D{static if} for something a bit more complicated than just dispatching between code paths: recursion\index{recursion}.  What if you know you will receive n-dimensional arrays\index{arrays} (simple arrays, arrays of arrays, arrays of arrays of arrays, \ldots), and want to use the fastest, super-optimized numerical function for the 1-dim array, another one for 2D arrays and yet another one for higher-level arrays. Abstracting this away, we need a template doing some introspection\index{compile-time!introspection} on types, that will return 0 for an element (anything that's not an array), 1 for a 1-dim array (\DD{T[]}, for some \DD{T}), 2 for a 2-dim array (\DD{T[][]}), and so on. Mathematicians call this the \emph{rank}\index{rank} of an array, so we will use that. The definition is perfectly recursive: 
\index{rank!definition}
\index{example!template recursion}
\index{template!recursion}
\begin{dcode}
template rank(T)
{
    static if (is(T t == U[], U)) // is T an array of U, for some type U?
        enum rank = 1 + rank!(U); // then let's recurse down.
    else                          // Base case, ending the recursion.
        enum rank = 0; 
}
\end{dcode}

Line 4 is the most interesting: with some \D{is} magic, \DD{U} has been determined by the compiler and is accessible inside the \D{static if} branch. We use it to peel one level of `\DD{[]}' off the type and recurse\index{recursion} downward, using \DD{U} as a new type for instantiating \DD{rank}. Either \DD{U} is itself an array (in which case the recursion will continue) or it will hit the base case and stop there.

\begin{dcode}
static assert(rank!(int)       == 0);
static assert(rank!(int[])     == 1);
static assert(rank!(int[][])   == 2);
static assert(rank!(int[][][]) == 3);

/* It will work for any type, obviously */
struct S {}

static assert(rank!(S)  == 0);
static assert(rank!(S[])== 1);
static assert(rank!(S*) == 0);
\end{dcode}

\aparte{\D{static assert}}{Putting \D{static} before an \D{assert}\index{static assert@\D{static assert}} forces the \D{assert} execution at compile-time. Using an \D{is} expression as the test clause gives assertion on types. 
One common use of \D{static assert} is to stop the compilation, for example if we ever get in a bad code path, by using \D{static assert}\DD{(0, someString)}. The string is then emitted as a compiler error message.}

\paragraph{Rank for Ranges:}\label{rankforranges}

D has an interesting sequence concept that's called \emph{range}\index{range} and comes with predefined testing templates in \std{range}. Why not extend \DD{rank}\index{rank} to have it deal with ranges and see if something is a range of ranges\index{range!of ranges} or more? A type can be tested to be a range with \DD{isInputRange} and its element type (the `\DD{U}') is obtained by applying \DD{ElementType} to the range type. Both templates are found in \std{range}. Also, since arrays are included in the range concept, we can entirely ditch the array part and use only ranges. Here is a slightly modified version of \DD{rank}:

\index{rank!for ranges}
\begin{dcode}
import std.range;

template rank(T)
{
    static if (isInputRange!T)                // is T a range?
        enum rank = 1 + rank!(ElementType!T); // if yes, recurse

    else                                      // base case, stop there
        enum rank = 0; 
}

auto c = cycle([[0,1],[2,3]]); // == [[0,1],[2,3],[0,1],[2,3],[0,1]...
assert(rank!(typeof(c)) == 2); // range of ranges 
\end{dcode}

\paragraph{Base Element Type:}\index{array!base element type}\index{range!base element type}

With \DD{rank}, we now have a way to get the number of \DD{[]}'s in an array type (\DD{T[][][]}) or the level of nesting in a range of ranges. The complementary query would be to get the base element type, \DD{T}, from any array of arrays \ldots of \DD{T} or the equivalent for a range. Here it is:

\begin{dcode}
template BaseElementType(T)
{
    static if (rank!T == 0)      // not a range
        static assert(0, T.stringof ~ " is not a range.");
    else static if (rank!T == 1) // simple range
        alias ElementType!T                     BaseElementType;
    else                         // at least range of ranges
        alias BaseElementType!(ElementType!(T)) BaseElementType;
}        
\end{dcode}

Line 4 is an example of \D{static assert} stopping compilation if we ever get into a bad code path. Line 8 is an example of a Matrioshka call: a template using another template's call as its parameter.

\paragraph{Generating Arrays:}

Now, what about becoming more generative by inversing the process? Given a type \DD{T} and a rank \DD{r} (an \D{int}), we want to obtain \DD{T[][]...[]}, with \DD{r} levels of \DD{[]}'s. A rank of 0 means producing \DD{T} as the result type.

\begin{dcode}
template NDimArray(T, int r)
{
    static if (r < 0)
        static assert(0, "NDimArray error: the rank must be positive.");
    else static if (r == 0)
        alias T NDimArray;
    else // r > 0
        alias NDimArray!(T, r-1)[] NDimArray;
}
\end{dcode}

Here, recursion\index{recursion} is done on line 8: we instantiate \DD{NDimArray!(T,r-1)}, which is a type, then create an array\index{arrays} of them by putting \DD{[]} at the end and expose it through an alias. This is also a nice example of using an integral value\index{template!parameters!integral value} as a template parameter.

\begin{dcode}
alias NDimArray!(double, 8) Level8;
static assert(is(Level8 == double[][][][][][][][]));
static assert(is(NDimArray!(double, 0) == double));
\end{dcode}

\paragraph{Repeated composition:} 

As a last example, we will use an alias template parameter\index{template!parameters!alias} in conjunction with some \D{static if} recursion\index{recursion}\index{static if@\D{static if}} to define a template that creates the `exponentiation' of a function, its repeated composition. Here is what I mean by this:

\begin{dcode}
string foo(string s) { return s ~ s;}

// power!(foo, n) is a function.
assert(power!(foo, 0)("a") == "a");                // identity function
assert(power!(foo, 1)("a") == foo("a"));           // "aa"
assert(power!(foo, 2)("a") == foo(foo("a")));      // "aaaa"
assert(power!(foo, 3)("a") == foo(foo(foo("a")))); // "aaaaaaaa"

// It's even better with function templates:
Arr[] makeArray(Arr)(Arr array) { return [array,array];}

assert(power!(makeArray, 0)(1) == 1);
assert(power!(makeArray, 1)(1) == [1,1]);
assert(power!(makeArray, 2)(1) == [[1,1],[1,1]]);
assert(power!(makeArray, 3)(1) == [[[1,1],[1,1]],[[1,1],[1,1]]]);
\end{dcode}

First, this is a template that `returns' (becomes, rather) a function. It's easy to do with the eponymous trick:\index{eponymous trick} just define inside the template a function with the same name. Secondly, it's clearly recursive in its definition, with two base cases: if the exponent is zero, then we shall produce the identity function and if the exponent is one, we shall just return the input function itself. That being said, \DD{power} writes itself:

\index{example!function composition}
\begin{dcode}
template power(alias fun, uint exponent)
{
    static if (exponent == 0) // degenerate case -> id function
        auto power(Args)(Args args) { return args; }
    else static if (exponent == 1) // end-of-recursion case -> fun
        alias fun power;
    else
        auto power(Args...)(Args args) 
        {
            return .power!(fun, exponent-1)(fun(args));
        }
}
\end{dcode}

\aparte{.power}{If you are wondering what's with the \DD{.power} syntax on line 10, it's because by defining an eponymous template, we hide the parent template's name. So inside \DD{power(Args...)} it refers to \DD{power(Args...)} and not \DD{power(}\D{alias}\DD{ fun, }\D{uint}\DD{ exponent)}. Here we want a new \DD{power} to be generated so we call on the global \DD{power} template with the `global scope'\index{scope!global scope operator}\index{operator!global scope operator (\DD{.})} operator (\DD{.}).}

In all three branches of \D{static if}, \DD{power} exposes a \DD{power} member, activating the eponymous template trick\index{eponymous trick} and allowing for an easy use by the client. Note that this template will work not only for unary (one argument) functions but also for n-args functions\footnote{ Except for the degenerate, $n = 0$ case, since the identity function defined above accepts only one arg. A more than one argument version is possible, but would need to return a tuple.}, for delegates and for structs or classes that define the \DD{()}(ie, \D{opCall}) operator\index{operator!opCall, ()@\DD{opCall}, \DD{()}} and for function templates\ldots\footnote{ I cheated a little bit there, because the resulting function accepts any number of arguments of any type, though the standard function parameters checks will stop anything untowards to happen. A cleaner (but longer, and for template functions, more complicated) implementation would propagate the initial function parameter typetuple.}

Now, are you beginning to see the power of templates?

\aparte{Curried Templates?}{\index{template!curried templates}No, I do not mean making them spicy, but separating the templates arguments, so as to call them in different places in your code. For \DD{power}, that could mean doing \D{alias }\DD{power!2 square;} somewhere and then using \DD{square!fun1}, \DD{square!fun2} at your leisure: the \DD{exponent} parameter and the \DD{fun} alias are separated. In fact, \DD{power} is already partially curried: \DD{fun} and \DD{exponent} are separated from \DD{Args}. For more on this, see section \ref{templatesintemplates}. 

Given a template \DD{temp}, writing a \DD{curry} template that automatically generates the code for a curried version of \DD{temp} is \emph{also} possible, but outside the scope of this document.\index{scope!outside the scope of this document}}


\subsection{Templates Specializations}\label{specializations}

\index{template!specialization}
Up to now, when we write a \DD{T} in a template parameter list, there is no constraint on the type that \DD{T} can become during instantiation. Template specialization is a small `subsyntax', restricting templates instantiations to a subset of all possible types and directing the compiler into instantiating a particular version of a template instead of another. If you've read Appendix \ref{isexpression} on the \D{is} expression, you already know how to write them. If you didn't, please do it now, as it's really the same syntax. These specializations are a direct inheritance from C++\index{C++} templates, up to the way they are written and they existed in D from the very beginning, long before \D{static if} or templates constraints were added. 

The specializations are added in the template parameter list, the \DD{(T, U, V)} part of the template definition. \index{syntax!templates specialization}\DD{Type : OtherType} restricts \DD{Type} to be implicitly convertible into \DD{OtherType} and \DD{Type == OtherType} restricts \DD{Type} to be exactly \DD{OtherType}. 

\index{example!templates specialization}
\begin{dcode}
template ElementType(T == U[], U) // can only be instantiated with arrays
{
    alias U ElementType;
}

template ElementType(T == U[n], U, size_t n) // only with static arrays
{
    alias U ElementType;
}

// Say Array is a class, which has an ElementType type alias defined.
template ElementType(T : Array) 
{
    alias Array.ElementType ElementType;
}
\end{dcode}

Now, the \DD{Type == AnotherType} syntax may seem strange to you: if you know you want to restrict \DD{Type} to be \emph{exactly} \DD{AnotherType}, why make it a template parameter? It's because of templates specializations' main use: you can write different implementations of a template (with the same name, obviously) and when asked to instantiate one of them, the compiler will automatically decide which one to use based the `most adapted' to the provided arguments. `Most adapted' obeys some complicated rules you can find on the D Programming Language web site, but they act in a natural way most of the time. The neat thing is that you can define the most general template \emph{and} some specialization. The specialized ones will be chosen when it's possible. For 

\index{example!templates specialization}
\begin{dcode}
template InnerType(T : U*, U) // Specialization for pointers
{
    alias U InnerType;
}

template InnerType(T : U[], U) // Specialization for dyn. arrays
{ ... }

template InnerType(T) // Standard, default case
{ ... }

int* p;
int i; 
alias InnerType!(typeof(p)) Pointer; // pointer spec. selected
alias InnerType!(typeof(i)) Default; // standard template selected
\end{dcode}

This idiom\index{idiom!template specialization} is used frequently in C++\index{C++}, where there is no (built-in) \D{static if} construct or template constraints. Oldish D templates used it a lot, too, but since other ways have been around for some years, recent D code seems to be more constraint-oriented: have a look at heavily templated Phobos modules, for example \std{algorithm} or \std{range}.

\aparte{Specializations or \D{static if} or Templates Constraints?}{Yes indeed. Let's defer this discussion for when we have seen all three subsystems.} 

\subsection{Default Values}\label{default}

Like functions paramters, templates parameters can have default values. The syntax is the same: \DD{Param = defaultValue}. The default can be anything that makes sense with respect to the parameter kind: a type, a literal value, a symbol or another template parameter.

\begin{dcode}
template Default(T = int, bool flag = false)
{ ... }

Default!(double);       // Instantiate Default!(double, false)
Default!(double, true); // Instantiate Default!(double, true) (Doh!)
Default!();             // Instantiate Default!(int, false)
\end{dcode}

A difference with function parameters default values is that, due to specializations (\ref{specializations}) or IFTI (\ref{ifti}), some parameters can be automatically deduced by the compiler. So, default templates parameters are not forced to be the last parameters:

\begin{dcode}
template Deduced(T : U[], V = U, U)
{ ... }

Deduced!(int[], double); // U deduced to be int. Force V to be a double.
Deduced!(int[]); // U deduced to be int. V is int, too.
Deduced!(); // Error, T is not some array.
\end{dcode}


\aparte{Specialization and Default Value?}{Yes you can. Put the specialization first, then the default value. Like this: \DD{(T : U[] = int[], U)}. It's not commonly used, though.}

As for functions, well-chosen defaults can greatly simplify standard calls. See for example the \stdanchor{algorithm}{sort}. It's parameterized on a predicate and a swapping strategy, but both are adapted to what most people need when sorting. That way, most client uses of the template will be short and clean, but customization to their own need is still possible.


\TODO{Maybe something on template dummy parameters, like those used by \stdanchor{traits}{ReturnType}. Things like \DD{dummy == }\D{void}.}

\section{Function Templates}\label{functiontemplates}

\index{template!function templates}
\subsection{Syntax}\label{functiontemplatessyntax}

If you come from languages with generics\index{generics}, maybe you thought D templates were all about parameterized classes and functions and didn't see any interest in the previous sections (acting on types?). Fear not, you can also do type-generic functions and such in D, with the added generative power of templates.

As we have seen in section \ref{eponymous}, if you define a function inside a template and use the template's own name, you can call it easily:

\index{syntax!function templates}
\begin{dcode}
// declaration:
template myFunc(T, int n)
{
    auto myFunc(T t) { return to!int(t) * n;}
}

// call:
auto result = myFunc!(double,3)(3.1415);

assert(result = to!int(3.1415)*3);
\end{dcode}

Well, the true story is even better. First, D has a simple way to declare a function template: just put a template parameter list before the argument list:

\index{syntax!function templates}
\index{example!function template}
\begin{dcode}
string concatenate(A,B)(A a, B b)
{
    return to!string(a) ~ to!string(b);
}

Arg select(string how = "max", Arg)(Arg arg0, Arg arg1)
{
    static if (how == "max")
        return (arg0 < arg1) ? arg1 : arg0;
    else static if (how == "min")
        return (arg0 < arg1) ? arg0 : arg1;
    else
        static assert(0, 
        "select: string 'how' must be either \"max\" or \"min\".");
}      
\end{dcode}

Nice and clean, uh? Notice how the return type can be templated too, using \DD{Arg} as a return type in \DD{select}. 

\subsection{\D{auto} return}\label{autoreturn}

Since you can select among code paths, the function return type can vary widely, depending on the template parameters you passed it. Use \D{auto} to simplify your code:\footnote{ \D{auto} return functions used to have some limitations, like for example not appearing in docs. I remember having to code type-manipulating templates to get correct the return type.}

\index{example!auto return@\D{auto} return}
\begin{dcode}
auto morph(T, U, alias f)(U arg)
{
    static if (is(U == int) && is(T == class))
    {
        return new T(f(arg));
    }
    else static if (is(T == function))
    {} // void-returning function
    else static if (...)
    (...)
}
\end{dcode}

\aparte{\D{auto ref}}{ A function template can have an \D{auto ref} return type. That means that for templates where the returned values are rvalues, the template will get the \D{ref}ed version. And the non-\D{ref} version if not.}

\subsection{IFTI}\label{ifti}

Even better is Implicit Function Template Instantiation (IFTI)\index{IFTI}\index{Implicit Function Template Instantiation}, which means that the compiler will most of the time be able to automatically determine a template parameters\index{template!parameters!IFTI} by studying the function arguments. If some template arguments are pure compile-time parameters, just provide them directly:

\begin{dcode}
/* suppose the previous concatenate(A,B) function */
string res1 = concatenate(1, 3.14); // A is int and B is double

struct Foo {}
string res2 = concatenate("abc", Foo()); // A is string, B is Foo

/* suppose the previous select(string how = "max", Arg) function */
auto res3 = select(3,4); // how is "max", Arg is int.
auto res4 = select!"min"(3.1416, 2.718); // how is "min", Arg is double.
\end{dcode}

As you can see, this results in very simple calling code. So we can both declare function templates and call them with a very clean syntax. The same can be done with structs or classes and such, as you will see in the next sections. In fact, the syntax is so clean that, if you are like me, you may forget from time to time that you are \emph{not} manipulating a function (or a struct, etc.): you are manipulating a template, a parameterized piece of code. 

\aparte{A Mantra}{\label{mantra}\index{mantra}\DD{XXX} templates are not \DD{XXX}s, they are templates. With \DD{XXX} being any of (function, struct, class, interface, union). Templates are parameterized scopes\index{scope!scopes are not first-class in D} and scopes are not first-class in D: they have no type, they cannot be assigned to a variable, they cannot be returned from functions. That means, for example, that you \emph{cannot} return function templates, you cannot inherit from class templates and so on. Of course, \emph{instantiated} templates are perfect examples of functions, classes, and such. Them you can inherit, return\ldots}

We may encounter The Mantra again in this tutorial.

\subsection{Example: Flattening Arrays and Ranges}\label{functionflatten}

Let's use what we have just seen in a concrete way. In D, you can manipulate 2D, 3D arrays\index{arrays}, but sometimes need to process them linearly. As of this writing, neither \std{algorithm} nor \std{range} provide a \DD{flatten} function. Beginning with simple arrays, here is what we want:

\begin{dcode}
assert( flatten([[0,1],[2,3],[4]]) == [0,1,2,3,4] );
assert( flatten([[0,1]]) == [0,1] );
assert( flatten([0,1]) == [0,1] );
assert( flatten(0) == 0 );

assert( flatten([[[0,1],[]], [[2]], [[3], [4,5]], [[]], [[6,7,8]]])
             == [0,1,2,3,4,5,6,7,8] );

assert( flatten([[[[[]]]]]) == [] ); // void[][][][][] -> void[]
\end{dcode}

So, studying the examples, we want a non-array or a simple array to be unaffected by \DD{flatten}: it just returns them. For arrays of rank\index{rank} 2 or higher, it collapses the elements down to a rank-1 array. It's classically recursive:\index{recursion} we will aply \DD{flatten} on all subarrays with \stdanchor{algorithm}{map} and concatenate the elements with \stdanchor{algorithm}{reduce}:

\begin{dcode}
import std.algorithm;

auto flatten(Arr)(Arr array)
{
    static if (rank!Arr == 0)
        return [array];
    else
    {
        auto children = map!(.flatten)(array);
        return reduce!"a~b"(children); // concatenate the children
    }
}
\end{dcode}

We make good use of D \D{auto} return parameter for functions there. In fact, a single call to \DD{flatten} will create one instance per level, all with a different return type. 

Note that \DD{flatten} works perfectly on ranges\index{range} too, but is not lazy: it eagerly concatenates all the elements down to the very last one in the innermost range. Ranges being lazy, a good \DD{flatten} implementation for them should itself be a range that delivers the elements one by one, calculating the next one only when asked to (and thus, would work on infinite or very long ranges too, which the previous simple implementation cannot do). Implementing this means creating a struct template (\ref{structtemplates}) with a factory function (\ref{factory}). You will find this as an example in section \ref{structflatten}.

From our current \DD{flatten}, it's an interesting exercise to add another parameter: the number of levels you want to flatten. Only the first three levels or last two innermost, for example. Just add an integral template parameter that gets incremented (or decremented) when you recurse and is another stopping case for the recursion. Positive levels could mean the outermost levels, while a negative argument would act on the innermost ones. A possible use would look like this:

\begin{dcode}
flatten!1([[[0,1],[]], [[2]], [[3], [4,5]], [[]], [[6,7,8]]]);
     // == [[[0,1],[],[2],[3], [4,5],[],[6,7,8]]
flatten!2([[[0,1],[]], [[2]], [[3], [4,5]], [[]], [[6,7,8]]]);
     // == [0,1,2,3,4,5,6,7,8]
flatten!0([[[0,1],[]], [[2]], [[3], [4,5]], [[]], [[6,7,8]]]);
     // ==[[[0,1],[]], [[2]], [[3], [4,5]], [[]], [[6,7,8]]]
flatten!(-1)([[[0,1],[]], [[2]], [[3], [4,5]], [[]], [[6,7,8]]]);
     // ==[[[0,1]], [[2]], [[3,4,5]], [[]], [[6,7,8]]]
\end{dcode}

\subsection{Anonymous Function Templates}\label{anonymousfunctions}

In D, you can define anonymous functions\index{anonymous!functions} (delegates even, that is: closures):

\begin{dcode}
auto adder(int a)
{
    return (int b) { return a+b;};
}

auto add1 = adder(1); // add1 is an int delegate(int)
assert(add1(2) == 3);
\end{dcode}

In the previous code, \DD{adder} returns an anonymous delegate. Could \DD{Adder} be templated? Ha! Remember The Mantra (page \pageref{mantra}):\index{mantra} function templates are templates and cannot be returned. For this particular problem, there are two possible solutions. Either you do not need any new type and just use \DD{T}:

\index{example!returning a function}
\begin{dcode}
auto adder(T)(T a)
{
    return (T b) { return a+b;};
}

auto add1f = adder(1.0) // add1f is an float delegate(float)
assert(add1f(2.0) == 3.0);

import std.bigint;

// addBigOne accepts a BigInt and returns a BigInt
auto addBigOne = adder(BigInt("1000000000000000");
assert(addBigOne(BigInt("1") == BigInt("1000000000000001");

// But:
auto error = add1(3.14); // Error! Waiting for an int, getting a double.
\end{dcode}

In the previous example, the returned anonymous delegate\index{anonymous!delegates} is \emph{not} templated. It just happens to use \DD{T}, which is perfectly defined once instantiation is done. If you really need to return something that can be called with any type, use an inner struct (see section \ref{innerstructs}).

Now, it may come as a surprise to you that D \emph{does} have anonymous function templates\index{function templates!anonymous function templates}\index{template!function templates!anonymous function templates}\index{anonymous!function templates}. The syntax is a purified version of anonymous functions:

\index{syntax!anonymous function templates}
\begin{dcode}
(a,b) { return a+b;} 
\end{dcode} 

Yes, the previous skeleton of a function is an anonymous template. But, remember The Mantra:\index{mantra} you cannot return them. And due to (to my eyes) a bug in \D{alias} grammar, you cannot alias them to a symbol:

\begin{dcode}
alias (a){ return a;} Id; // Error
\end{dcode}

So what good are they? You can use them with template alias parameters, when these stand for functions and function templates:

\index{example!anonymous function template}
\begin{dcode}
template callTwice(alias fun)
{
    auto callTwice(T)(T t)
    {
        return fun(fun(t));
    }
}

alias callTwice!( (a){ return a+1;}) addTwo;

assert(addTwo(2) == 4);
\end{dcode}

Since they are delegates, they can capture local symbols:

\begin{dcode}
enum b = 3; // Manifest constant, initialized to 3

alias callTwice!( (a){ return a+b;}) addTwoB;

assert(addTwoB(2) == 2 + 3 + 3);
\end{dcode}

\subsection{Functions Overloading}\label{functionsoverloading}

\TODO{Write something on this.}

\subsection{Storage Classes}\label{storageclasses}

As seen in section \ref{instantiating}, storage classes get applied to types during instantiation.\index{template!parameters!storage class} It also works for function templates arguments:

\index{example!storage class!function parameters}
\begin{dcode}
void init(T)(ref T t)
{ 
    t = T.init;
}

int i = 10;
init(i);
assert(i == 0);
\end{dcode}

Should the need arises, that also means you can customize your storage classes according to templates arguments. There is no built-in syntax for that, so you'll have to resort to our good friend \D{static if}\index{static if@\D{static if}!functions storage classes} and the eponymous trick:\index{eponymous trick!functions storage classes}

\index{example!storage class}
\index{is expression@\D{is} expression!type specialization}
\index{static if@\D{static if}!nested}
\begin{dcode}
// Has anyone a better example?
template init(T)
{
    static if (is(T == immutable) || is(T == const))
        void init(T t) {} // do nothing
    else static if (is(T == class))
        void init(ref T t)
        {
            t = new T();
        }
    else 
        void init(ref T t)
        {
            t = T.init;
        }
}
\end{dcode}

\subsection{Properties are Automatically Deduced}\label{autodeduce}

In D, a function can have the following properties:

\begin{itemize}
\item A function can be tagged with the \D{pure} property, which means it does not have side-effects: the value you get back is the only thing that matters. 
\item They can also be tagged with \D{@safe}, \D{@trusted} and \D{@system}. \D{@safe} means a function cannot corrupt memory. A \D{@trusted} function can call \D{@safe} ones, but offers no other garanty concerning memory. And a \D{@system} function just laugh at you.
\item The last property is \D{nothrow} which means you guarantee the function does not throw any exception.
\end{itemize}

As the compiler gets complete access to a function template code, it can analyze it and automatically deduce properties for you. This feature is still quite new as of this writing, but it does seem it works. So, \emph{all} of your function templates will get a smattering of properties when they are instantiated (these properties will of course vary with the template parameters).

\subsection{\D{in} and \D{out} Clauses}\label{inandoutclauses}

The \D{in} and \D{out} clauses for a function are given full access to templates parameters. As for other parameterized code, that means you can use \D{static if} to enable or disable code, depending on the template arguments.

\index{example!in and out clauses@\D{in} and \D{out} clauses}
\index{example!static if@\D{static if}!enabling/disabling code}
\begin{dcode}
import std.complex, std.math, std.traits;

auto squareRoot(N n) if (isNumeric!N || isComplex!N)
in 
{
    static if (isNumeric!N)
        enforce(n > 0);
    // no need to do that for a complex.
}
body
{
    return sqrt(n);
}
\end{dcode}

\subsection{Modifying Functions: Memoizing a Function} \label{memoizing}

\TODO{Should this section be moved in the struct templates section?}

Let us use a template to wrap a function and provide some additional usefulness when calling the function. Memoizing is an interesting and useful example: if the function does long calculations, it might be efficient to store the computed results in an external structure and to query this structure for the result instead of calling the function again.

We have not seen struct templates yet (they are presented in section \ref{structtemplates}), but the following example should be easy to understand: the previous result are stored in an associative array, indexed on tuples of arguments. To get a function return type or parameter type tuple, just use Phobos' \stdanchor{traits}{ReturnType} and \stdanchor{traits}{ParameterTypeTuple}, which are templates that accept function \emph{names} or types.

\index{struct templates!memoize@\DD{memoize}}
\index{function templates!wrapping a function!memoize@\DD{memoize}}
\index{template!parameters!alias}
\begin{dcode}
struct Memoize(alias fun)
{
    alias ReturnType!fun RT;
    alias ParameterTypeTuple!fun PTT;
    RT[Tuple!(PTT)] memo; // stores the result, indexed by arguments.

    RT opCall(PTT args) {
        if (tuple(args) in memo) {    // Have we already seen these args?
            return memo[tuple(args)]; // if yes, use the stored result
        }
        else {                        // if not, compute the result and store it.
            RT result = fun(args);
            memo[tuple(args)] == result;
            return result;
        }
    }
}

Memoize!fun memoize(alias fun)()
{
    return Memoize!fun();
}
\end{dcode}

Usage is very simple:

\begin{dcode}
int veryLongCalc(int i double d, string s) { ... }

auto vlcMemo = memoize!(veryLongCalc);

// calculate veryLongCalc(1, 3.1r4, "abc")
// takes minutes!
int res1 = vlcMemo(1, 3.14, "abc"); 
int res2 = vlcMemo(2, 2.718, "def");// minutes again!
int res3 = vlcMemo(1, 3.14, "abc"); // a few ms to get res3
\end{dcode}

The above code is trivial and could be optimized in many ways. Mostly, a real memoizing template should also modify its behavior with storing policies. For example:

\begin{itemize}
\item No-limit or limited size store? 
\item In case of limited-size store: how to define the limit and what should be the eviction policy?
\begin{itemize}
\item First-in/First-out memo?
\item Least recenly used memo?
\item Least used?
\item Time-to-live?
\item Discard all and flush the store?
\item Discard only a fraction?
\item Stop memoizing?
\end{itemize}
\end{itemize}

The last X results could be stored in a queue: each time a result is pushed into the associative array, push the arguments tuples in the queue. Once you reach the maximum store limit, discard the oldest one or (for example) half the stored values.

Here is a possible small implementation. It makes for a nice example of enabling/disabling code with \D{static if} and \D{enum}-based policies. Note that I use D dynamic arrays as a primitive queue. A real queue could probably be more efficient, but there isn't one in the standard library as of this writing.

\index{struct templates!memoize@\DD{memoize}}
\index{function templates!wrapping a function!memoize@\DD{memoize}}
\index{template!parameters!alias}
\index{example!enabling/disabling code}
\index{example!enum-based policy@\D{enum}-based policy}
\index{idiom!enabling/disabling code}
\index{idiom!policy}
\begin{dcode}
enum MemoStoringPolicy { 
    always,  // there is no tomorrow
    maximum  // sustainable growth
}

enum MemoDiscardingPolicy { 
    oldest,   // only discard the oldest result
    fraction, // discard a fraction (0.5 == 50%)
    all       // burn, burn!
}

struct Memoize(alias fun, 
               MemoStoringPolicy storing,
               MemoDiscardingPolicy discarding)
{
    alias ReturnType!fun RT;
    alias ParameterTypeTuple!fun PTT;

    static if (storing == MemoStoringPolicy.maximum)
    {
        Tuple!(PTT)[] argsQueue;
        size_t maxNumStored;
    }
    
    static if (discarding == MemoDiscardingPolicy.fraction)
        float fraction;

    RT[Tuple!(PTT)] memo; // stores the result, indexed by arguments.
    
    RT opCall(PTT args) {
        if (tuple(args) in memo) {    // Have we already seen these args?
            return memo[tuple(args)]; // if yes, use the stored result
        }
        else {                        // if not, 
            RT result = fun(args);    // compute the result and store it.
            memo[tuple(args)] == result;
            return result;
        }
    }
}
\end{dcode}

And a few factory function to help creating those \DD{Memoize} structs:

\begin{dcode}
// No runtime arg -> always store
Memoize!(fun, MemoStoringPolicy.always, MemoDiscardingPolicy.all)
memoize(alias fun)()
{
    return Memoize!(fun, 
                    MemoStoringPolicy.always, 
                    MemoDiscardingPolicy.all)();
}

// One runtime size_t arg -> maximum store / discarding all
Memoize!(fun, MemoStoringPolicy.maximum, MemoDiscardingPolicy.all)
memoize(alias fun)(size_t max)
{
    return Memoize!(fun, 
                    MemoStoringPolicy.maximum, 
                    MemoDiscardingPolicy.all)(max);
}

// Two runtime args (size_t, double) -> maximum store / discarding a fraction
Memoize!(fun, MemoStoringPolicy.maximum, MemoDiscardingPolicy.fraction)
memoize(alias fun)(size_t max, double fraction)
{
    return Memoize!(fun, 
                    MemoStoringPolicy.maximum, 
                    MemoDiscardingPolicy.fraction)(max, fraction);
}

// One compile-time argument (discarding oldest), one runtime argument (max)
Memoize!(fun, MemoStoringPolicy.maximum, discarding)
memoize(alias fun, MemoDiscardingPolicy discarding == MemoDiscardingPolicy.oldest)
(size_t max)
{
    return Memoize!(fun, 
                    MemoStoringPolicy.maximum, 
                    discarding)(max);
}
\end{dcode}

Most of the time, the type of runtime arguments is enough to determine what you want as a memoizing/storing behavior. Only for the (rarer?) policy of discarding only the oldest stored result does the user need to indicate it with a template argument:

\begin{dcode}
int veryLongCalc(int i double d, string s) { ... }

// Store the first million results, flush the memo on max
auto vlcMemo1 = memoize!(veryLongCalc)(1_000_000);

// Store the first million results, flush half the memo on max
auto vlcMemo2 = memoize!(veryLongCalc)(1_000_000, 0.5f);

// Store first twenty results, discard only the oldest
auto vlcMemo3 = memoize!(veryLongCalc, MemoDiscardingPolicy.oldest)(20);
\end{dcode}

\subsection{Modifying Functions: Currying a Function} \label{currying}

\index{static if@\D{static if}!recursion}
\index{static if@\D{static if}!nested}
\index{template!double-decker template!CheckCompatibility.With@\DD{CheckCompatibility.With}}
\begin{dcode}
template CheckCompatibility(T...)
{
    template With(U...)
    {
        static if (U.length != T.length)
            enum With = false;
        else static if  (T.length == 0) // U.length == 0 also
            enum With = true;
        else static if (!is(U[0] : T[0]))
            enum With = false;
        else
            enum With = CheckCompatibility!(T[1..$]).With!(U[1..$]);
    }
}
\end{dcode}

\index{operator!opCall, ()@\DD{opCall}, \DD{()}}
\index{template!parameters!integral value}
\index{template!parameters!alias}
\index{static assert@\D{static assert}}
\begin{dcode}
struct Curry(alias fun, int index = 0)
{
    alias ReturnType!fun RT;
    alias ParameterTypeTuple!fun PTT;
    PTT args;

    auto opCall(V...)(V values)
        if (V.length > 0
         && V.length + index <= PTT.length)
    {
        // Is fun directly callable with the provided arguments?
        static if (__traits(compiles, fun(args[0..index], values)))
            return fun(args[0..index], values);
        // If not, the new args will be stored. We check their types.
        else static if (!CheckCompatibility!(PTT[index..index + V.length]).With!(V))
            static assert(0, "curry: bad arguments. Waited for "
                            ~ PTT[index..index + V.length].stringof
                            ~ " but got " ~ V.stringof);
        // not enough args yet. We store them.
        else
        {
            Curry!(fun, index+V.length) c;
            foreach(i,a; args[0..index]) c.args[i] = a;
            foreach(i,v; values) c.args[index+i] = v;
            return c;
        }
    }
}

auto curry(alias fun)()
{
    Curry!(fun,0) c;
    return c;
}
\end{dcode}

\section{Struct Templates}\label{structtemplates}
\subsection{Syntax}\label{structsyntax}

As you might have guessed, declaring a struct template is done like this:

\index{syntax!struct templates}
\index{example!struct template}
\begin{dcode}
struct Tree(T)
{
    T value;
    Tree[] children;

    bool isLeaf() @property { return children.empty;}
    /* More tree functions: adding children, removing some,... */
}     
\end{dcode}

\aparte{\DD{Tree[]} or \DD{Tree!(T)[]}?}{Remember that inside a template declaration, the template's name refers to the current instantiation. So inside \DD{Tree(T)}, the name \DD{Tree} refers to a \DD{Tree!T}.}

This gives us a run-of-the-mill generic tree, which is created like any other template:

\begin{dcode}
auto t0 = Tree!int(0);
auto t1 = Tree!int(1, [t0,t0]);
Tree!int[] children = t1.children;
\end{dcode}

As with all previous templates, you can parameterize your structs using much more than simple types:

\index{example!struct template}
\begin{dcode}
bool lessThan(T)(T a, T b) { return a<b;}

struct Heap(Type, alias predicate = lessThan, float reshuffle = 0.5f)
{
    // predicate governs the internal comparison
    // reshuffle deals with internal re-organizing of the heap
    Type[] values;
(...)
}
\end{dcode}

Struct templates are heavily used in \std{algorithm} and \std{range} for lazy iteration, have a look there.

\subsection{Factory Functions}\label{factory}

Now, there is one limitation: structs constructors do not do activate IFTI (\ref{ifti}) like template functions do. In the previous subsection to instantiate \DD{Tree(T)} I had to explicitly indicate \DD{T}:

\begin{dcode}
auto t0 = Tree!int(0); // Yes.

auto t1 = Tree(0); // Error, no automatic deduction that T is int.
\end{dcode}

This is because templated constructors are possible (see \ref{constructortemplates}) and may have template parameters differing from that of the global struct template. But honestly that's a pain, even more so for struct templates with many template arguments. There is a solution, of course: use a template function to create the correct struct and return it. Here is an example of such a factory function for \DD{Tree}:

\index{example!factory function}
\begin{dcode}
auto tree(T)(T value, T[] children = null) 
{
    return Tree!(T)(value, children);
}

auto t0 = tree(0); // Yes!
auto t1 = tree(1, [t0,t0]); // Yes!

static assert(is( typeof(t1) == Tree!int ));

auto t2 = tree(t0); // Yes! typeof(t2) == Tree!(Tree!(int)) 
\end{dcode}

Once more, have a look at \std{algorithm} and \std{range}, they show numerous examples of this idiom.\index{idiom!factory function}

\subsection{Giving Access to Inner Parameters}\label{givingaccess}

As was said in section \ref{inneralias}, template arguments are not accessible externally once the template is instantiated. For the \DD{Tree} example, you might want to get an easy access to \DD{T}. As for any other templates, you can expose the parameters by aliasing them. Let's complete our \DD{Tree} definition:

\index{example!struct template!inner alias parameters}
\begin{dcode}
struct Tree(T)
{
    alias T Type;
    T value;
    Tree[] children;

    bool isLeaf() @property { return children.empty;}
}     

auto t0 = tree("abc");
alias typeof(t0) T0;

static assert(is( T0.Type == string ));
\end{dcode}

\subsection{Templated Member Functions}\label{membertemplates}

A struct template is a template like any other: you can declare templates inside, even functions templates. Which means you can have templated member functions. 

\paragraph{Mapping on a Tree} Let us use that to give our \D{Tree} a mapping ability. For a range,\index{range} you can use \stdanchor{algorithm}{map} to apply a function in turn to each element, thus delivering a transformed range. The same process can be done for a tree, thereby keeping the overall \emph{shape} but modifying the elements.\footnote{ We could easily make that a free function, but this \emph{is} the member function section.}

Let's think a little bit about it before coding. \DD{map} should be a function template that accepts any function name as a template alias parameter\index{template!parameters!alias} (like \stdanchor{algorithm}{map}). Let's call this alias \DD{fun}. The \DD{value} member should be transformed by \DD{fun}, that's easy to do. We want to return a new \DD{Tree}, which will have for type parameter the result type of \DD{fun}. If \DD{fun} transforms \DD{A}s into \DD{B}s, then a \DD{Tree!A} will be mapped to a \DD{Tree!B}. However, since \DD{fun} can be a function template, it may not have a pre-defined return type that could be obtained by \stdanchor{traits}{ReturnType}. We will just apply it on a \DD{T} value (obtained by \DD{T.init} and take this type. So \DD{B} will be \DD{typeof(fun(T.init))}.

What about the children? We will map \DD{fun} on them too and collect the result into a new children array. They will have the same type: \DD{Tree!(B)}. If the mapped \DD{Tree} is a leaf (ie: if it has no children), the process will stop.

Since this is a recursive template,\index{recursion} we have to help the compiler a bit with the return type. Here we go:\footnote{ The difference with Phobos\index{Phobos} \DD{map} is that it's not lazy.}

\index{example!templated method}
\begin{dcode}
/* rest of Tree code */
    Tree!(typeof(fun(T.init))) map(alias fun)()
    {
        alias typeof(fun(T.init)) MappedType;
        MappedType mappedValue = fun(value);
        Tree!(MappedType)[] mappedChildren;
        foreach(child; children) mappedChildren ~= child.map!(fun);
        return tree(mappedValue, mappedChildren);
    }
\end{dcode}

Let's use it:

\begin{dcode}
auto t0 = tree(0);
auto t1 = tree(1, [t0,t0]);
auto t2 = tree(2, [t1, t0, tree(3)]);

/* t2 is       2
             / | \
            1  0  3
           / \
          0   0
*/

// t2 is a Tree!(int)
static assert(is( t2.Type == int ));

// Adding one to all values
int addOne(int a) { return a+1;}
auto t3 = t2.map!(add(1));

/* t3 is       3
             / | \
            2  1  4
           / \
          1   1
*/

assert(t3.value = 3);

// Converting all values to strings
import std.conv:to;
auto ts = t2.map!(to!string); // we convert every value into a string;

/* ts is      "2"
             / | \
           "1""0""3"
           / \
         "0"  "0"
*/

assert(is( ts.Type == string ));
assert(ts.value == "2");
\end{dcode}

\paragraph{Folding a Tree} You may feel \DD{map} is not really a member function: it does not take any argument. Let's make another transformation on \DD{Tree}s: folding them, that is: collapsing all values into a new one. The range equivalent is \stdanchor{algorithm}{reduce} which collapses an entire (linear) range\index{range} into one value, be it a numerical value, another range or what have you

For a tree, folding can for example generate all values in pre-order or post-order, calculate the height of the tree, the number of leaves\ldots As for ranges, folding is an extremely versatile function. It can in fact be used to convert a \DD{Tree} into an array or another \DD{Tree}. We will do just that.

Taking inspiration from \DD{reduce}, we need an seed value and \emph{two} folding functions. The first one, \DD{ifLeaf}, will be called on childless nodes, for \DD{fold} to return \DD{ifLeaf(value, seed)}. The second one, \DD{ifBranch}, will be called nodes with children. In this case, we first apply \DD{fold} on all children and then return \DD{ifBranch(value, foldedChildren)}. In some simple cases, we can use the same function, hence a default case for \DD{ifBranch}. Here is the code:\footnote{ Technically, \stdanchor{algorithm}{reduce} is a \emph{left fold}, while what is shown here is a \emph{right fold}. The difference is not essential to this document.} 

\index{example!templated method}
\begin{dcode}
/* rest of Tree code */
auto fold(alias ifLeaf, alias ifBranch = ifLeaf, S)(S seed)
{
    if (isLeaf)
    {
        return ifLeaf(value, seed);
    }
    else
    {
        typeof(Tree.init.fold!(ifLeaf, ifBranch)(seed))[] foldedChildren;
        foreach(child; children) 
            foldedChildren ~= child.fold!(ifLeaf, ifBranch)(seed);
        return ifBranch(value, foldedChildren);
    }
}
\end{dcode}

Let's play a bit with it. First, we want to sum all values of a tree. For leaves, we just return the node's \DD{value} plus \DD{seed}. For branches, we are given \DD{value} and an array containing the sums of values for all children. We need to sum the values of this array, add it to the node's \DD{value} and return that. In that case, we do not care for the seed.

\begin{dcode}
auto sumLeaf(T, S)(T value, S seed)
{
    return value + seed;
}

auto sumBranch(T)(T value, T[] summedChildren)
{
    return value + reduce!"a+b"(summedChildren);
}

auto sum = t2.fold!(sumLeaf, sumBranch)(0);
assert(sum == 2 + (1 + 0 + 0) + 0 + 3);
\end{dcode}

In the same family, but a bit more interesting is getting all values for in-order iteration: given a tree node, return an array containing the local value and then the values for all nodes, recursively.

\begin{dcode}
T[] inOrderL(T, S)(T value, S seed)
{
    return [value] ~ seed;
}

T[] inOrderB(T)(T value, T[][] inOrderChildren)
{
    return [value]  ~  reduce!"a~b"(inOrderChildren);
}

int[] seed; // empty array
auto inOrder = t2.fold!(inOrderL, inOrderB)(seed);
assert(inOrder == [2, 1, 0, 0, 0, 1, 3]);
\end{dcode}

And as a last use, why not build a tree? 

\TODO{Write just that.}

\subsection{Templated Constructors}\label{constructortemplates}

Struct constructors are member function, so they can be templated too. They need not have the same template parameters than their mother struct:

\index{syntax!templated constructors}
\begin{dcode}
struct S(T)
{
    this(U)(U u) { ... }
}

auto s = S!string(1); // T is string, U is int.
\end{dcode}

As you can see, IFTI (\ref{ifti})\index{IFTI!for constructors} works for constructors. \DD{U} is automatically deduced, though you have to indicate \DD{T} in this case. The previous example is drastically limited: you cannot have any value of type \DD{U} in the struct, because \DD{U} does not exist outside the constructor. A bit more useful would be to collect an alias (a function, for example) and use it to initialize the struct. If it's used only for initialization, it can be discarded afterwards. But then, IFTI is not activated by an alias\ldots

The most interesting use I've seen is to make conversions during the struct's construction:

\index{example!templated constructor}
\begin{dcode}
struct Holder(Type)
{
    Type value;

    this(AnotherType)(AnotherType _value)
    {
        value = to!Type(_value);
    }
}

Holder!int h = Holder!int(3.14);
assert(h.value == 3);
\end{dcode}

That way, a \DD{Holder!int} can be constructed with any value, but if the conversion is possible, it will always hold an \D{int}.

\subsection{Inner Structs}\label{innerstructs}

You can create and return inner structs and use the local template parameters in their definition. We could have a factory function for \DD{Heap} like this:

\index{example!inner struct}
\index{template!parameters!alias}
\begin{dcode}
auto heap(alias predicate, Type)(Type[] values)
{
    struct Heap
    {
        Type[] values;
        this(Type[] _values)
        {
            /* some code initializing values using predicate
        }
        /* more heapy code */
    }

    return Heap(values); // alias predicate is implicit there
}
\end{dcode}

In that case, the \DD{Heap} struct is encapsulated inside \DD{heap} and uses the \DD{predicate} alias inside its own engine, but it's not a templated struct itself. I did not use \DD{Tree} as an example, because with recursive types it becomes tricky.

By the way, strangely enough, though you cannot declare `pure' templates inside functions, you can declare struct templates. Remember the \DD{adder} function in section \ref{anonymousfunctions}? It didn't need to be templated with one type for each argument, as most of the time when you add numbers, they have more or less the same type. But what about a function that converts its arguments to \D{string}s before concatenating them?

\begin{dcode}
auto concatenate(A)(A a)
{
    /* !! Not legal D code !!
    return (B)(B b) { return to!string(a) ~ to!string(b);}
}
\end{dcode}

The previous example is not legal D code.\index{mantra} Of course, there is a solution: just return a struct with a template member function (\ref{membertemplates}), in that case the \DD{opCall} operator:

\index{example!inner struct}
\index{example!templated methods}
\index{example!opCall, ()@\DD{opCall}, \DD{()}}
\index{operator!opCall, ()@\DD{opCall}, \DD{()}}
\begin{dcode}
auto concatenate(A)(A a)
{
    struct Concatenator 
    {
        A a;

        auto opCall(B)(B b)
        {
            return to!string(a) ~ to!string(b);
        }
    }

    Concatenator c;
    c.a = a; // So as not to activate opCall()

    return c;
}

auto c = concatenate(3.14);
auto cc = c("abc");
assert(cc == "3.14abc");
\end{dcode}

See section \ref{operatoroverloading} for more on operators overloading.

What about templated inner structs inside struct templates? It's perfectly legal:

\index{example!inner struct!templated}
\begin{dcode}
struct Outer(O)
{
    O o;

    struct Inner(I)
    {
        O o;
        I i;
    }

    auto inner(I)(I i) { return Inner!(I)(o,i);}
}

auto outer(O)(O o) { return Outer!(O)(o);}

auto o = outer(1); // o is an Outer!int;
auto i = o.inner("abc"); // Outer.Outer!(int).Inner.Inner!(string)
}
\end{dcode}

\subsection{Template This Parameters}\label{this}

Inside a struct or class template, there is another kind of template parameter: the template this parameter,\index{template!parameters!template this parameter} declared with \D{this} \DD{identifier}. \DD{identifier} then gets the type of the \D {this} reference. It's useful mainly for two uses:

\begin{itemize}
\item mixin templates (\ref{mixintemplates}), where you do not know the enclosing type beforehand. Please see this section for some examples.
\item to determine how the type of the \D{this} reference is qualified (\D{const}, \D{immutable}, \D{shared}, \D{inout} or unqualified).
\end{itemize}

\TODO{Some kind of example?}

\subsection{Example: a Concat / Flatten Range}\label{structflatten}

We will use what we've seen for a struct template  a lazy range\index{range} that flattens ranges of ranges\index{range!of ranges} into linear ranges. Remember the \DD{flatten} function from section \ref{functionflatten}? It worked quite well but was \emph{eager}, not \emph{lazy}: given an infinite range (a cycle, for example) it would choke on it. We will here make a lazy flattener.

If you look at the ranges defined in \std{range}, you will see that most (if not all) of them are structs. That's the basic way to get laziness in D: the struct holds the iteration state and exposes the basic range primitives. At the very least to be an \emph{input range} --- the simplest kind of range ---, a type must have the following members\index{range!members} (be they properties, member functions or manifest constants):

\begin{description}
\item[front:] Which returns the range's first element.
\item[popFront:] Which discards the first element and advances the range by one step.
\item[empty:] Which returns \D{true} if the range has no more element, \D{false} otherwise.
\end{description}

From this simple basis, powerful algorithms can be designed that act on ranges. D defines more refined range concepts by adding other constraints. A \emph{forward range} adds the \DD{save} member that's used to store a range internal state and allows an algorithm to start again from a saved position. A \emph{bidirectional range} also has the \DD{back} and \DD{popBack} primitives for accessing the end of the range, and so on.

Here we will begin by creating a simple input range\index{range} that takes a range of ranges\index{range!of ranges} and iterate on the inner elements. Let's begin with the very basis:

\index{example!range}
\index{example!struct template}
\index{example!factory function}
\begin{dcode}
import std.range;

struct Flatten(Range)
{
    Range range;
    ...
}

auto flatten(Range)(Range range)
{ 
    static if (rank!Range == 0)
        static assert(0, "flatten needs a range.");
    else static if (rank!Range == 1)
        return range;
    else
        return Flatten!(Range)(range);
}
\end{dcode}

So we have a struct template and its associated factory function. It doesn't make sense to instantiate \DD{Flatten} with any old type, so \DD{Range} is checked to be a range, using the \DD{rank}\index{range} template we saw on page \pageref{rankforranges}. We haven't seen template constraints yet (they are described in section \ref{constraints}), but they would be a good fit there too.

A range of ranges\index{range!of ranges} can be represented like this:

\begin{dcode}
[ subRange1[elem11, elem12,...]
, subRange2[elem21, elem22,...] 
, ... ]
\end{dcode}

We want \DD{Flatten} to return elements in this order: \DD{elem11, elem12, \ldots  elem21, elem22, \ldots}. Note that for ranges of rank higher than 2, the \DD{elemxy}s are themselves ranges. At any given time, \DD{Flatten} is working on a sub-range, iterating on its elements and discarding it when it's empty. The iteration will stop when the last subrange has been consumed, that is when \DD{range} itself is empty.

\index{example!struct template}
\index{example!range}
\index{example!inner member alias}
\index{range}
\begin{dcode}
struct Flatten(Range)
{
    alias ElementType!Range    SubRange;
    alias ElementType!SubRange Element;

    Range range;
    SubRange subRange;

    this(Range _range) {
        this.range = range;
        discardEmptySubRanges();
    },

    Element front() { return subRange.front;}

    bool empty() { return range.empty;}

    void popFront() {
        if (!subRange.empty) subRange.popFront;
        discardEmptySubRanges();
    }

    void discardEmptySubRanges() {
        while(subRange.empty && !range.empty) {
            range.popFront;
            if (!range.empty) subRange = range.front;
        }
    }
}
\end{dcode}

\begin{itemize}
\item I cheat a little bit with D standard bracing style, because it eats vertical space like there is no tomorrow.
\item We begin on line 3 and 4 by defining some new types used by the methods. They are not strictly necessary but make the code easier to understand and expose these types to the outer world, if they are needed.
\item A constructor is now necessary to correctly initialize the struct.
\item \DD{front} returns a subrange element.
\item The \DD{discardEmptySubRanges} function does what it says on the can: we do not iterate on empty subranges.
\end{itemize}

Let see if this template works, by creating an infinite range and giving it to \D{flatten}:

\begin{dcode}
import std.range;

auto cy = cycle(["Hello", "World"); // "Hello","World","Hello","World",...
auto flattened = flatten(cy);
assert(flattened.front == 'H');

auto takeTwelve = take(flattened, 12);
assert(array(takeTwelve) == "HelloWorldHe");
\end{dcode}

But then, this only works for ranges of ranges\index{rank!of ranges} (of rank $\leq$ 2). We want something that flattens ranges of any rank down to a linear range. This is easily done, we just add recursion\index{recursion} in the factory function:\index{factory function}

\index{example!factory function}
\index{example!recursion}
\index{example!static if@\D{static if}}
\begin{dcode}
auto flatten(Range)(Range range)
{ 
    static if      (rank!Range == 0)
        static assert(0, "flatten needs a range.");
    else static if (rank!Range == 1)
        return range;
    else static if (rank!Range == 2)
        return Flatten!(Range)(range);
    else         // rank 3 or higher
        return flatten(Flatten!(Range)(range));
}
\end{dcode}

And, testing:

\index{IFTI}
\index{example!IFTI}
\begin{dcode}
auto rank3 = [[[0,1,2],[3,4,5],[6]]
             ,[[7],[],[8,9],[10,11]]
             ,[[],[]]
             ,[[12]]  ];

auto flat = flatten(rank3);
assert(rank!(typeof(flat)) == 1); // Yup, it's a linear range
assert(equal( flat, [0,1,2,3,4,5,6,7,8,9,10,11,12] ));

auto reallyFlat = flatten(flat);
assert(equal( reallyFlat, flat )); // No need to insist

import std.string, std.algorithm;

auto text =
"Sing, O goddess, the anger of Achilles son of Peleus,
 that brought countless ills upon the Achaeans.
 Many a brave soul did it send hurrying down to Hades,
 and many a hero did it yield a prey to dogs and vultures,
 for so were the counsels of Jove fulfilled
 from the day on which the son of Atreus, king of men,
 and great Achilles, first fell out with one another.";
auto lines = text.splitLines;   // array of strings
string[][] words;
foreach(line; lines) words ~= array(splitter(line, ' '));
assert( rank!(typeof(words)) == 3); // range of range of strings
                                    // range of range of array of chars
auto flat = flatten(words);

assert(equal(take(flat, 50), 
             "Sing,Ogoddess,theangerofAchillessonofPeleus,thatbr"));
\end{dcode}

Here it is. It works and we used a struct template (this section, \ref{structtemplates}), \D{static if} (\ref{staticif}), inner member alias (\ref{inneralias}), factory functions (\ref{factory}) and IFTI (\ref{ifti}).

\section{Class Templates}\label{classtemplates}

\aparte{This Section Needs You!}{I'm not an OOP programmer and am not used to create interesting hierarchies. If anyone reading this has an interesting example of class templates that could be used throughout the section, I'm game.}

\subsection{Syntax}\label{classsyntax}

No surprise there, just put the template parameters list between \D{class} and the optional inheritance indication.

\index{syntax!class templates}
\begin{dcode}
class MyClass(Type, alias fun, bool b = false)
    : Base, Interface1, Interface2
{ ... }
\end{dcode}

What's more fun is that you can have parameterized inheritance: the various template parameters are defined before the base class list, you can use them here:

\index{syntax!class templates!parameterized inheritance}
\begin{dcode}
class MyClass(Type, alias fun, bool b = false)
    : Base!(Type), Interface1, Interface2!(fun,b)
{ ... }
\end{dcode}

\aparte{Interface Templates?}{Yes you can. See section \ref{othertemplates}}

This opens interesting vistas, where what a class inherits is determined by its template arguments (since \DD{Base} may be many different classes or even interfaces depending on \DD{Type}). In fact, look at this:

\begin{dcode}
enum WhatBase { Object, Interface, BaseClass }

template Base(WhatBase whatBase = WhatBase.Object)
{
    static if (is(T == WhatBase.Object))
        alias Object Base; // MyClass inherits directly from Object
    else static if(is(T == WhatBase.Interface))
        alias TheInterface Base;
    else
        alias TheBase Base;
}
\end{dcode}

With this, \DD{MyClass} can inherit either from \DD{Object}, D root to the class hierarchy, or from an interface or from another class. Obviously, the dispatching template could be much more refined. With a second template parameter, the base class could itself be parameterized, and so on.

What this syntax \emph{cannot} do however is change the number of interfaces at compile-time.\footnote{ Except, maybe, by having an interface template be empty for certain parameters, thus in effect disappearing from the list.} It's complicated to say: `with \emph{this} argument, \DD{MyClass} will inherit from \DD{I}, \DD{J} and \DD{K} and with \emph{that} argument, it will inherit only from \DD{L}.' You'd need the previous interfaces to all participate in the action, to all be templates and such. If the needed interfaces are all pre-defined and not templated, you need wrapping templates. It's a pain. However, type tuples can be used to greatly simplify this (see section \ref{inheritancelist} for an example).

\subsection{Methods Templates}\label{methodtemplates}

An object's methods are nothing more than delegates with a reference to the local \D{this} context. As seen for structs (\ref{membertemplates}), methods can be templates too.

\TODO{Something on overriding methods with templates.}
\TODO{Find some interesting method example. Man, I do not do classes.}

\subsection{\D{invariant} clauses}\label{class:invariant}

In the same family than \D{in}/\D{out} clauses for functions (section \ref{inandoutclauses}), a class template's \D{invariant} clause has access to the template parameter. You cannot make it disappear totally, but you can get it to be empty with a \D{static if} statement.

\index{example!class template!invariant}
\index{static if@\D{static if}!enabling/disabling code!invariant clause@\D{invariant} clause}
\begin{dcode}
class MyClass(T, U, V)
{
    (...)

    invariant
    {
        static if (/* some condition on T,U,V */ )
        {
            /* invariant code */
        }
        else
        { /* empty invariant */ }
    }
}
\end{dcode}

\subsection{Inner Classes}\label{innerclasses}

It's the same principle than for structs (\ref{innerstructs}). You can define inner classes using the template parameters. You can even give them methods templates that use other template arguments. There is really nothing different from inner structs.

\subsection{Anonymous Classes}\label{anonymousclasses}

In D, you can return anonymous classes\index{anonymous!classes} directly from a function or a method. Can these be templated? Well, they cannot be class templates, that wouldn't make sense.\index{mantra} But you can return anonymous classes with templated methods, if you really need to.

\index{example!anonymous class templates}
\index{anonymous!class templates}
\begin{dcode}
// stores a function and a default return value.
auto acceptor(alias fun, D)(D default)
{
    return new class 
    { 
        auto opCall(T)(T t)
        {
            static if (__traits(compiles, fun(T.init)))
                return fun(t); 
            else
                return default;
        }
    };
}

int add1(int i) { return i+1;}
auto accept = acceptor!(add1)(0);
auto test1 = accept(1); 
assert(test1 == 2);

auto test2 = accept("abc");
assert(test2 == 0); // default value
\end{dcode}

\TODO{Test this!}

For \D{\_\_traits}\DD{(compiles, ...)},\index{traits@\D{\_\_traits}} see \href{www.dlang.org/traits.html}{here}.

\subsection{Parameterized Base Class}

You can use a template parameter directly as a base class:

\index{example!parameterized base class}
\index{template!parameters!as a base class}
\begin{dcode}
interface ISerializable
{
    size_t serialize() @property;
}

class Serializable(T) : T, ISerializable
{
    size_t serialize() @property { ... }
} 
\end{dcode}

In this example, a \DD{Serializable!SomeClass} can act as a \DD{SomeClass}. It's not different from what you would do with normal classes except the idiom\index{idiom!parameterized base class} is now abstracted on the base class: you write the template once, it can then be used on any class.

If you have different interfaces like this, you can nest these properties:

\index{example!nested templates instantiations}
\begin{dcode}
auto wrapped = new Serializable!(Iterable!(SomeClass))(...);
\end{dcode}

Of course, the base class and the interface may themselves be parameterized:

\begin{dcode}
enum SerializationPolicy { policy1, policy2 }

interface ISerializable
(SerializationPolicy policy = SerializationPolicy.policy1)
{
    static if (is(policy == SerializationPolicy.policy1))
        ...
    else
        .../
}

class Serializable(T, Policy) : T, ISerializable!Policy
{
    ...
} 
\end{dcode}

In D, you can also get this kind of effect with an \D{alias X this;} declaration in your class or struct. You should also have a look at mixin templates (\ref{mixintemplates}) and wrapper templates (\ref{wrappertemplates}) for other idioms built around the same need. 

\subsection{Another Example}

\begin{dcode}
/**
Timon Gehr timon.gehr@gmx.ch via puremagic.com 
This is an useful pattern. I don't have a very useful example at hand, but this one should do. It does similar things that can be achieved with traits in Scala for example.
*/
import std.stdio;
abstract class Cell(T){
       abstract void set(T value);
       abstract const(T) get();
private:
       T field;
}

class AddSetter(C: Cell!T,T): C{
       override void set(T value){field = value;}
}
class AddGetter(C: Cell!T,T): C{
       override const(T) get(){return field;}
}

class DoubleCell(C: Cell!T,T): C{
       override void set(T value){super.set(2*value);}
}

class OneUpCell(C: Cell!T,T): C{
       override void set(T value){super.set(value+1);} 
}

class SetterLogger(C:Cell!T,T): C{
       override void set(T value){
               super.set(value);
               writeln("cell has been set to '",value,"'!");
       }
}

class GetterLogger(C:Cell!T,T): C{
       override const(T) get(){
               auto value = super.get();
               writeln("'",value,"' has been retrieved!");
               return value;
       }
}

class ConcreteCell(T): AddGetter!(AddSetter!(Cell!T)){}
class OneUpDoubleSetter(T): OneUpCell!(DoubleCell!(AddSetter!(Cell!T))){}
class DoubleOneUpSetter(T): DoubleCell!(OneUpCell!(AddSetter!(Cell!T))){}
void main(){
       Cell!string x;
       x = new ConcreteCell!string;
       x.set("hello");
       writeln(x.get());

       Cell!int y;
       y = new SetterLogger!(ConcreteCell!int);
       y.set(123); // prints: "cell has been set to '123'!
       
       y = new GetterLogger!(DoubleCell!(ConcreteCell!int));
       y.set(1234);
       y.get(); // prints "'2468' has been retrieved!"

       y = new AddGetter!(OneUpDoubleSetter!int);
       y.set(100);
       writeln(y.get()); // prints "202"

       y = new AddGetter!(DoubleOneUpSetter!int);
       y.set(100);
       writeln(y.get()); // prints "201"

       // ...
}
\end{dcode}

\subsection{The Curiously Recurring Template Pattern}

\begin{dcode}
class Base(Child) { ... }

class Derived : Base!Derived { ... }
\end{dcode}

Hold on, what does that mean? \DD{Base} is easy to understand. But what about \DD{Derived}? It says it inherits from another class that is templated on\ldots \DD{Derived} \emph{itself}? But \DD{Derived} is not defined at this stage! Or is it? 
Yes, that works. It's called CRTP, which stands for Curiously Recurring Template Pattern\index{Curiously Recurring Template Pattern} (see \href{ http://en.wikipedia.org/wiki/Curiously_recurring_template_pattern}{Wikipedia} on this). But what could be the interest of such a trick?

As you can see in the Wikipedia document it's used either to obtain a sort of compile-time binding or to inject code in your derived class. For the latter, D offers mixin templates (\ref{mixintemplates}) which you should have a look at. CRTP comes from C++\index{C++!CRTP} where you have multiple inheritance. In D, I fear it's not so interesting. Feel free to prove me wrong, I'll gladly change this section.

\subsection{Example}

\TODO{An example with duplicator, a template that takes a class and creates another class which is its clone: same base, same interfaces, etc.}

\TODO{What about an expression template?}

\section{Other Templates?}\label{othertemplates}

Two other aggregate types in D can be templated using the same syntax: interfaces and unions.

\subsection{Interface Templates}

The syntax is exactly what you might imagine: 

\index{syntax!interface templates}
\begin{dcode}
interface Interf(T)
{
    T foo(T);
    T[] bar(T,int);
}
\end{dcode}

Templated interfaces are sometimes useful but as they look very much like class templates, I won't describe them. As before, remember The Mantra (see page \pageref{mantra}\index{mantra}): interface templates are \emph{not} interfaces.

\subsection{Union Templates}

Here is the syntax, no suprise there:

\index{syntax!union templates}
\begin{dcode}
union Union(A,B,C) { A a; B b; C c;}
\end{dcode}

Union templates seem like a good idea, but honestly I've never seen one. Any reader of this document, please give me an example if you know one. 

Strangely, enumerations do not have the previous simplified syntax. To declare a templated enumeration, use the eponymous template trick:

\index{syntax!enum template}
\index{example!enum template}
\begin{dcode}
template Enum(T)
{
    enum Enum : T { A, B, C}
}
\end{dcode}
