\newpage
\phantomsection
\part*{Introduction}\label{intro} %%%%%%%%%%%%%%%%%%%%%%%%%%%%%%%%%%%%%%%%%%%%%%%%%%
\addcontentsline{toc}{part}{Introduction}

Templates are a central feature of D, giving you powerful compile-time code generation abilities that'll make your code cleaner, more flexible and even more efficient. They are used everywhere in \href{http://www.dlang.org/phobos/}{Phobos}\index{Phobos} --- D standard library --- and therefore any D user should know about them. But, based on C++\index{C++}'s templates as they are, D templates can be a bit daunting at first. The \href{http://www.dlang.org}{D Programming Language} website's \href{http://www.dlang.org/template.html}{documentation} is a good start, though its description of templates is spread among many different files and (as it's a language reference) its material doesn't so much \emph{teach} you how to use templates as \emph{show} you their syntax and semantics.

This document aims to be a kind of tutorial on D templates, to show the beginning D coder what can be achieved with them. When I was doing C++\index{C++}, I remember \emph{never} using templates for more than \emph{containers-of-T} stuff, and considered Boost-level\footnote{ The \href{http://www.boost.org}{Boost}\index{Boost} C++ library collection makes heavy use of templates.} metaprogramming the kind of code I could never understand, never mind produce. Well, D's sane syntax for templates and nifty features such as \D{static if}, \D{alias} or tuples cured me of that impression. I hope this document will help you, too.

\section*{What's in This Document}
\addcontentsline{toc}{section}{What's in This Document}

\autoref{basics} deals with the very basics: how to declare and instantiate a template, the standard `building blocks' you'll use in almost all your templates, along with function (\ref{functiontemplates}), struct (\ref{structtemplates}) and class (\ref{classtemplates}) templates. Throughout the text, examples will present applications of these concepts. 

\autoref{advanced} is about more advanced topics a D template user will probably use, but not on a daily basis, like template constraints (\ref{constraints}), mixin templates (\ref{mixintemplates}) or operator overloading (\ref{operatoroverloading}). 

\autoref{around} presents other meta\-pro\-gram\-ming tools: string mixins (\ref{stringmixins}), compile-time function evaluation (\ref{ctfe}), and \D{\_\_traits} (\ref{traits}). These are seen from a \mbox{template-y} point of view: how they can interact with templates and what you can build with them in conjunction with templates.

\autoref{examples} presents more developed examples of what can be done with templates, based on real needs I had at some time and that could be fulfilled with templates.

Finally, an appendix (\autoref{isexpression}) on the ubiquitous \D{is} expression  completes this document.

\section*{Conventions}
\addcontentsline{toc}{section}{Conventions}

In this doc:

\begin{itemize}
\item D keywords will be marked like this: \D{int}, \D{static if}, \D{\_\_traits}.
\item Symbols and names used in code samples and cited in the text will be written like this: \DD{myFunc}, \DD{flatten}.
\item internal links will be in red, like this (\ref{basics}).
\item external links will be in blue, like \href{http://www.dlang.org}{this}.
\end{itemize}

Syntax-highlighted code samples are shown like this:

\begin{ndcode}
import std.stdio;

// This is a comment
void hello(uint times)
{
    foreach(i; 0..times) writeln("Hello, Word!");
}
\end{ndcode}

I'll use numbered lines only when necessary. The code lines may be wider than the main text, but that's to get about 80-chars-long lines. For the time being I'll just respect the standard \LaTeX\ text width, unless many people complain about it.

For those of you interested by \LaTeX, this document uses the \href{http://code.google.com/p/minted/}{minted}\index{minted (LaTex package)@\DD{minted} (\LaTeX\ package)} package for the code samples. 

I will sometimes make a little aparte, discussing a small piece of info too small to be in its own section. These will be marked so:

\aparte{What's with all that red and blue?}{What, don't you like having red boxes and blue keywords in your text? Even links will be colored!}

Finally, some sections in this doc are not finished yet. The sections I consider unfinished will begin with a friendly frame:

\unfinished{That's also true for this very intro. I should have a 'Thanks' subsection where I list people that contributed to this text.}

\section*{How to Get This Document}
\addcontentsline{toc}{section}{How to Get This Document}

This doc is just a bunch of \LaTeX documents \href{http://github.com/PhilippeSigaud/D-templates-tutorial}{hosted on GitHub}. Don't hesitate to fork it or (even better for me) to make pull requests! For those of you reading this on paper, the address is:

\vspace{8pt}
\url{http://github.com/PhilippeSigaud/D-templates-tutorial}

