\newpage
\part{Examples}\label{examples}

\section{Type Sorcery}

\subsection{Mapping, Filtering and Folding Types}

\TODO{Say something about mapping with a typetuple-returning template}
\begin{verbatim}
template Qual(T)
{
    alias TypeTuple!(T,T[],const(T)[],immutable(T)[], shared(T)[], 
const T, immutable T, shared T, const T[], immutable T[], shared T[]) Qual;
}

alias TypeTuple!(bool,ubyte,byte,ushort,short,uint,int,ulong,long,float,double,real,char,wchar,dchar) 
ValueTypes;

alias staticMap!(Qual,ValueTypes) QualifiedTypes;
\end{verbatim}

\subsection{Interspersing Types, Crossing Types}

\begin{verbatim}
/**
Helper template. Given T0, T1, T2, ..., Tn, Tn+1, ... T2n, will returns
the interleaving of the first part with the second part: T0, Tn+1, T1, Tn+2, ... Tn, T2n
It's fragile: no test, etc. A better way to do this would be as a two-steps template: Interleave!(T...).With!(U...)
*/
template Interleave(T...)
{
    static if (T.length > 1)
        alias TypeTuple!(T[0], T[$/2], Interleave!(T[1..$/2], T[$/2+1 .. $])) Interleave;
    else
        alias T Interleave;
}
\end{verbatim}

\section{Relational Algebra}

Inspiration for this example comes from \href{http://david.rothlis.net/d/templates}{This blog article}.

\TODO{Extracting from a tuple: project, select. Also, natural/inner/outer join, cartesian product. And intersection/union/difference. rename!( "oldField", "newField"). Databases are just dynamic arrays of tuples.}.

\section{Cloning, sort of}

\TODO{(Elsewhere) creating a class from a struct?}

\section{Recording Successive States}

From Andrej Mitrovic.

\begin{verbatim}
import std.stdio;
import std.traits;

struct Shape
{
    int x, y;
    void foo(int val) { x += val; }
    int bar(int val) { y += val; return y; }
}

struct RecorderImpl(T)
{
    T t;
    T[] t_states;
    
    this(T t)
    {
        this.t = t;
        t_states ~= t;
    }
    
    auto opDispatch(string method, Args...)(Args args)
    {
        static if (mixin("is(ReturnType!(t." ~ method ~ ") == void)"))
        {
            mixin("t." ~ method ~ "(args);");
            t_states ~= t;
        }
        else
        {
            mixin("auto result = t." ~ method ~ "(args);");
            t_states ~= t;
            return result;
        }
    }
    
    auto opIndex(size_t index)
    {
        assert(index < t_states.length);
        return t_states[index];
    }
    
    auto opSlice(size_t lowerBound, size_t upperBound)
    {
        return t_states[lowerBound..upperBound];
    }
    
    auto opSlice()
    {
        return t_states[];
    }
}

auto Recorder(T)(T t)
{
    return RecorderImpl!T(t);
}

void main()
{
    auto shape = Recorder(Shape(0, 0));
    
    shape.foo(5);
    shape.bar(5);
    
    writeln(shape[0]);
    writeln(shape[]);
}
\end{verbatim}



\section{Fields}

From Jacob's Carlborg Orange.

\TODO{Test typeof(s.tupleof)}

\begin{verbatim}
/**
 * Evaluates to an array of strings containing the names of the fields in the given type
 */
template fieldsOf (T)
{
	const fieldsOf = fieldsOfImpl!(T, 0);
}

/**
 * Implementation for fieldsOf
 * 
 * Returns: an array of strings containing the names of the fields in the given type
 */
template fieldsOfImpl (T, size_t i)
{
	static if (T.tupleof.length == 0)
		const fieldsOfImpl = [""];

	else static if (T.tupleof.length - 1 == i)
		const fieldsOfImpl = [T.tupleof[i].stringof[1 + T.stringof.length + 2 .. $]];

	else
		const fieldsOfImpl = T.tupleof[i].stringof[1 + T.stringof.length + 2 .. $] ~ fieldsOfImpl!(T, i + 1);
}
\end{verbatim}

\section{Extending an enum}

\begin{verbatim}
string EnumDefAsString(T)() if (is(T == enum)) {
  string result = "";
  foreach (e; __traits(allMembers, T)) {
      result ~= e ~ " = T." ~ e ~ ",";
  }
  return result;
}

template ExtendEnum(T, string s) if (is(T == enum) &&
is(typeof({mixin("enum a{"~s~"}");}))) {
  mixin("enum ExtendEnum {" ~
      EnumDefAsString!T() ~ s ~
  "}");
}

unittest {
  enum bar {
      a = 1,
      b = 7,
      c = 19
  }

  import std.typetuple;

  alias ExtendEnum!(bar, q{ // Usage example here.
      d = 25
  }) bar2;

  foreach (i, e; __traits(allMembers, bar2)) {
      static assert( e == TypeTuple!("a", "b", "c", "d")[i] );
  }

  assert( bar2.a == bar.a );
  assert( bar2.b == bar.b );
  assert( bar2.c == bar.c );
  assert( bar2.d == 25 );

  static assert(!is(typeof( ExtendEnum!(int, "a"))));
  static assert(!is(typeof( ExtendEnum!(bar, "25"))));
}

//(Simen Kjaeraas)
\end{verbatim}

\section{Static Switching} \label{examples:staticswitch}

\TODO{What, no compile-time switch? Let's create one}.
Example of: tuples, type filtering (in constraints), recursion, etc.

\begin{verbatim}
template staticSwitch(List...) // List[0] is the value commanding the switching
                               // It can be a type or a symbol.
{
    static if (List.length == 1) // No slot left: error
        static assert(0, "StaticSwitch: no match for " ~ List[0].stringof);
    else static if (List.length == 2) // One slot left: default case
        enum staticSwitch = List[1];
    else static if (is(List[0] == List[1]) // Comparison on types
                || (  !is(List[0])         // Comparison on values
                   && !is(List[1])
                   && is(typeof(List[0] == List[1]))
                   && (List[0] == List[1])))
        enum staticSwitch = List[2];
    else
        enum staticSwitch = staticSwitch!(List[0], List[3..$]);
}
\end{verbatim}

\section{Gobble}\label{gobble}

\begin{verbatim}

struct Gobbler(T...)
{
    T store;
    Gobbler!(T, string,U) opBinary(string op, U)(U u) if (op == "~")
    {
        return Gobbler!(T,string, U)(store, op, u);
    }
}

Gobbler!() gobble() { return Gobbler!()();}
\end{verbatim}

\section{A Polymorphic Tree}\label{polymorphictree}

\section{Polymorphic Association Lists}\label{associationlists}

Usage: a bit like Lua tables: structs, classes (you can put anonymous functions in them?),  namespaces.
Also, maybe to add metadata to a type?

\begin{verbatim}
template Half(T...)
{
    static if (T.length <= 1)
        alias TypeTuple!() Half;
    else
        alias TypeTuple!(T[0], Half!(T[2..$])) Half;
}

struct AList(T...)
{
    static if (T.length >= 2 && T.length % 2 == 0)
        alias Half!T Keys;
    else static if (T.length >= 2 && T.length % 2 == 1)
        alias Half!(T[0..$-1]) Keys;
    else
        alias TypeTuple!() Keys;

    static if (T.length >= 2)
        alias Half!(T[1..$]) Values;
    else
        alias TypeTuple!() Values;

    template at(alias a)
    {
        static if ((staticIndexOf!(a, Keys) == -1) && (T.length % 2 == 1)) // key not found, but default value present
            enum at = T[$-1]; // default value
        else static if ((staticIndexOf!(a, Keys) == -1) && (T.length % 2 == 0))
            static assert(0, "AList: no key equal to " ~ a.stringof);
        else //static if (Keys[staticIndexOf!(a, Keys)] == a)
            enum at = Values[staticIndexOf!(a, Keys)];
    }
}

alias AList!( 1,     "abc"
            , 2,     'd'
            , 3,     "def"
            , "foo", 3.14
            ,        "Default") al;

writeln("Keys: ", al.Keys);
writeln("Values: ", al.Values);
writeln("at!1: ", al.at!(1));
writeln("at!2: ", al.at!(2));
writeln("at!\"foo\": ", al.at!("foo"));
writeln("Default: ", al.at!4);
\end{verbatim}

\section{Expression Templates}\label{expressiontemplates}

\section{Wrapping a Function}

Making it accept a tuple, for example.

\begin{verbatim}
template tuplify(alias fun)
{
    auto tuplify(T...)(Tuple!T tup)
    {
        return fun(tup.expand);
    }
}
\end{verbatim}

Another interesting (and much more complicated) example is \DD{juxtapose}.

\section{Mapping n ranges in parallel}\label{parallelmapping}

\begin{verbatim}
// Very easy to do, now:
auto nmap(alias fun, R...)(R ranges) if (allSatisfy!(isInputRange,R))
{
    return map!(tuplify!fun)(zip(ranges));
}
\end{verbatim}

More complicated: \DD{std.algorithm.map} accepts to take more than one function as template argument. In that case, the functions are all mapped in parallel on the range, internally using \DD{std.functional.adjoin}.
Here we can extend \DD{nmap} to accept $n$ functions in parallel too. There is a first difficulty:

\begin{verbatim}
auto nmap(fun..., R...)(R ranges) if (allSatisfy!(isInputRange, R))
{ ... 
\end{verbatim}

See the problem? Tuples must be the last parameter of a template: there can be only one. Double-stage templates come to the rescue:

\begin{verbatim}
template nmap(fun...) if (fun.length >= 1)
{
    auto nmap(R...)(R ranges) if (allSatisfy!(isInputRange, R))
    {...}
}
\end{verbatim}

Final code:

\begin{verbatim}
template nmap(fun...) if (fun.length >= 1)
{
    auto nmap(R...)(R ranges) if (allSatisfy!(isInputRange, R))
    {
        alias adjoin!(staticMap!(tuplify, fun)) _fun;
        return map!(_fun)(zip(ranges));
    }
}
\end{verbatim}

Give an example with \DD{max}, it works!

And here is the $n$-ranges version of \DD{std.algorithm.filter}:

\begin{verbatim}
auto nfilter(alias fun, R...)(R ranges) if (allSatisfy!(isInputRange, R))
{
    return filter!(tuplify!fun)(zip(ranges));
}
\end{verbatim}

\section{Statically-Checked Writeln}

\section{Extending a Class}\label{extendingaclass}

There is regularly a wish in the D community for something called Universal Function Call Syntax (UFCS):\index{syntax!Universal Function Call Syntax}\index{UFCS} the automatic transformation of \DD{a.foo(b)} into \DD{foo(a,b)} when \DD{a} has no member called \DD{foo} and there \emph{is} a free function called \DD{foo} in the local scope\index{scope!local scope}. This already works for arrays\index{arrays!UFCS} (hence, for strings) but not for other types.

There is no way to get that in D for built-in types except by hacking the compiler, but for user-defined types, you can call templates to the rescue.

\DD{opDispatch} can be used to forward to an external free function. A call \D{this}\DD{.method(a,b)} becomes \DD{method(}\D{this}\DD{,a,b)}.

\begin{verbatim}
mixin template Forwarder
{
    auto opDispatch(string name, Args...)(Args args)
    {
        mixin("return " ~ name ~ "(args);");
    }
}
\end{verbatim}

In D, a void \D{return} clause is legal: 

\begin{verbatim}
return;
// or
return void;
\end{verbatim}

So if \DD{name(}\D{this}\DD{,a,b)} is a \D{void}-returning function, all is OK.


\section{Pattern Matching With Functions}

\section{Generating a Switch for Tuples}
Case 0:, etc.

\section{Tuples as Sequences}

\subsection{Mapping on Tuples}

\subsection{Filtering Tuples}

\begin{verbatim}
(1, "abc", 2, "def", 3.14)
->
((1,2),("abc","def"),(3,14))
\end{verbatim}

